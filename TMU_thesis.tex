%        نمونه پایان‌نامه آماده شده با استفاده از کلاس TarbiatModares، نگارش 0.4
%        سجاد نوریان، دانشگاه تربیت مدرس،   
%-----------------------------------------------------------------------------------------------------
%        اگر قصد نوشتن پروژه کارشناسی را دارید، در خط زیر به جای msc، کلمه bsc و اگر قصد نوشتن رساله دکتری
%        را دارید، کلمه phd را قرار دهید. کلیه تنظیمات لازم، به طور خودکار، اعمال می‌شود.
% !TEX TS-program = xelatex
\documentclass[oneside,openany,12pt,msc]{TarbiatModares}
%       فایل commands.tex را حتماً به دقت مطالعه کنید؛ چون دستورات مربوط به فراخوانی بسته زی‌پرشین 
%       و دیگر بسته‌ها و ... در این فایل قرار دارد و بهتر است که با نحوه استفاده از آنها آشنا شوید.

% در این فایل، دستورها و تنظیمات مورد نیاز، آورده شده است.
%--------------------------------------------------------------Tarbiat Modares-----------------------------------------------------
\usepackage{algpseudocode}

\usepackage{multirow}
\usepackage{rotating}
\usepackage{lscape}
\usepackage{float}

% در ورژن جدید زی‌پرشین برای تایپ متن‌های ریاضی، این سه بسته، حتماً باید فراخوانی شود
\usepackage{amsthm,amssymb,amsmath}
% بسته‌ای برای تنطیم حاشیه‌های بالا، پایین، چپ و راست صفحه
\usepackage[top=40mm, bottom=30mm, left=25mm, right=35mm]{geometry}
\setlength{\textwidth}{15cm}
\setlength{\oddsidemargin}{2mm}
\renewcommand{\baselinestretch}{1.5}     % ꉑ¬‰Ü‰‚ ¡‰Î‰Ï‰
% بسته‌‌ای برای ظاهر شدن شکل‌ها و تصاویر متن
\usepackage{graphicx}
% بسته‌ای برای رسم کادر
%\usepackage{framed} 
% بسته‌‌ای برای چاپ شدن خودکار تعداد صفحات در صفحه «معرفی پایان‌نامه»
\usepackage{lastpage}
% بسته‌ و دستوراتی برای ایجاد لینک‌های رنگی با امکان جهش
\usepackage[linktocpage=true,colorlinks,pagebackref=true,linkcolor=blue,citecolor=magenta]{hyperref}
%\usepackage[pagebackref=false,colorlinks,linkcolor=blue,citecolor=magenta]{hyperref}
% چنانچه قصد پرینت گرفتن نوشته خود را دارید، خط بالا را غیرفعال و  از دستور زیر استفاده کنید چون در صورت استفاده از دستور زیر‌‌، 
% لینک‌ها به رنگ سیاه ظاهر خواهند شد که برای پرینت گرفتن، مناسب‌تر است
%\usepackage[pagebackref=false]{hyperref}
% بسته‌ لازم برای تنظیم سربرگ‌ها
\usepackage{fancyhdr}
% بسته‌ای برای ظاهر شدن «مراجع» و «نمایه» در فهرست مطالب
\usepackage[nottoc]{tocbibind}
% دستورات مربوط به ایجاد نمایه
\usepackage{makeidx}
\makeindex
%%%%%%%%%%%%%%%%%%%%%%%%%%
% فراخوانی بسته زی‌پرشین و تعریف قلم فارسی و انگلیسی
\usepackage{xepersian}
\settextfont[Scale=1.1]{XB Niloofar.ttf}

%%%%%%%%%%%%%%%%%%%%%%%%%%
% چنانچه می‌خواهید اعداد در فرمول‌ها، انگلیسی باشد، خط زیر را غیرفعال کنید
%\setdigitfont[Scale=0.9]{Persian Modern}
%\setdigitfont.ttf[Scale=1.1]{XB Zar}
%\setdigitfont[Scale.ttf=1.1]{B Nazanin Outline}
%%%%%%%%%%%%%%%%%%%%%%%.ttf%%%
% تعریف قلم‌های فارسی و انگلیسی اضافی بر.ttfای استفاده در بعضی از قسمت‌های متن
\defpersianfont\nastaliq[Scale=2.02]{IranNastaliq.ttf}
\defpersianfont\bnazaninout[Scale=2]{B Nazanin Outline.ttf} 
%\defpersianfont\chapternumber[Scale=2.5]{XB Kayhan Pook.ttf}
%\renewcommand{\chapternumber}[1][Scale=2.5]{\fontspec{XB Kayhan Pook.ttf}}
\defpersianfont\dav[Scale=2]{XB Kayhan Pook.ttf}
\defpersianfont\titr[Scale=1.02]{XB Titre.ttf}
\defpersianfont\chapternumber[Scale=3]{XB Niloofar.ttf}
%\renewcommand{\chapternumber}[1][Scale=3]{\fontspec{XB Niloofar.ttf}}

%%%%%%%%%%%%%%%%%%%%%%%%%%
% دستوری برای حذف کلمه «چکیده»
%\renewcommand{\abstractname}{}
% دستوری برای حذف کلمه «abstract»
\renewcommand{\latinabstract}{}
% دستوری برای تغییر نام کلمه «اثبات» به «برهان»
\renewcommand\proofname{\textbf{برهان}}
% دستوری برای تغییر نام کلمه «کتاب‌نامه» به «مراجع»
\renewcommand{\bibname}{کتاب‌نامه}
\def\contentsname{فهرست مندرجات}
\def\listfigurename{لیست اشکال}
% دستوری برای تعریف واژه‌نامه انگلیسی به فارسی
\newcommand\persiangloss[2]{#1\dotfill\lr{#2}\\}
% دستوری برای تعریف واژه‌نامه فارسی به انگلیسی 
\newcommand\englishgloss[2]{#2\dotfill\lr{#1}\\}
% تعریف دستور جدید «\پ» برای خلاصه‌نویسی جهت نوشتن عبارت «پروژه/پایان‌نامه/رساله»
\newcommand{\پ}{پروژه/پایان‌نامه/رساله }
%%%%%%%%%%%%%%%%%%%%%%%%%%
% تعریف و نحوه ظاهر شدن عنوان قضیه‌ها، تعریف‌ها، مثال‌ها و ...
\theoremstyle{definition}
\newtheorem{definition}{تعریف}[section]
\theoremstyle{theorem}
\newtheorem{theorem}[definition]{قضیه}
\newtheorem{lemma}[definition]{لم}
\newtheorem{proposition}[definition]{گزاره}
\newtheorem{corollary}[definition]{نتیجه}
\newtheorem{remark}[definition]{ملاحظه}
\theoremstyle{definition}
\newtheorem{example}[definition]{مثال}
\newcommand{\x}{\mbox{\boldmath$x$\unboldmath}}
\newcommand{\dd}{\mbox{\boldmath$d$\unboldmath}}
\newcommand{\Uu}{\mbox{\boldmath$U$\unboldmath}}
 \newcommand{\balpha}{\mbox{\boldmath$\alpha$\unboldmath}}
 \newcommand{\bbeta}{\mbox{\boldmath$\beta$\unboldmath}}
  \newcommand{\btheta}{\mbox{\boldmath$\theta$\unboldmath}}
 \newcommand{\bphi}{\mbox{\boldmath$\phi$\unboldmath}}
  \newcommand{\bpsi}{\mbox{\boldmath$\psi$\unboldmath}}
\newcommand{\bmu}{\mbox{\boldmath$\mu$\unboldmath}}
 \newcommand{\X}{\mbox{\boldmath$X$\unboldmath}}
\newcommand{\Y}{\mbox{\boldmath$Y$\unboldmath}}
\newcommand{\z}{\mbox{\boldmath$z$\unboldmath}}
 \newcommand{\W}{\mbox{\boldmath$W$\unboldmath}}
 \newcommand{\q}{\mbox{\boldmath$q$\unboldmath}}
 \newcommand{\bb}{\mbox{\boldmath$b$\unboldmath}}
 \newcommand{\I}{\mbox{\boldmath$I$\unboldmath}}
\newcommand{\w}{\mbox{\boldmath$w$\unboldmath}}
\newcommand{\tb}{\mbox{\boldmath$\theta$\unboldmath}}
\newcommand{\bSigma}{\mbox{\boldmath$\Sigma$\unboldmath}}
\newcommand{\bsigma}{\mbox{\boldmath$\sigma$\unboldmath}}
\newcommand{\logit}{{\rm{logit}}}
\newcommand{\rth}{{\rm{th}}}
\newcommand{\E}{{\rm{E}}}
\newcommand{\Var}{{\rm{Var}}}
\newcommand{\Pq}{{\rm{P}}}
\newcommand{\ICHS}{مطالعه سلامت کودکان اندونزیایی}
\newcommand{\OR}{{\rm{OR}}}
\newcommand{\Corr}{{\rm{Corr}}}
\newcommand{\Cov}{{\rm{Cov}}}
%
%%%%%%%%%%%%%%%%%%%%%%%%%%%%
% دستورهایی برای سفارشی کردن سربرگ صفحات
\csname@twosidefalse\endcsname
\pagestyle{fancy}
\fancyhf{} 
\fancyhead[RE]{\rightmark}
\fancyhead[LO]{\leftmark}
\fancyhead[LE,RO]{\thepage}
\renewcommand{\chaptermark}[1]{%
\markboth{\thechapter.\ #1}{}}
% دستورهایی برای سفارشی کردن سربرگ صفحات
%\csname@twosidetrue\endcsname
%\pagestyle{fancy}
%\fancyhf{} 
%\fancyhead[OL,EL]{\thepage}
%\fancyhead[OR]{\small\rightmark}
%\fancyhead[ER]{\small\leftmark}
%\renewcommand{\chaptermark}[1]{%
%\markboth{\thechapter.\ #1}{}}
%%%%%%%%%%%%%%%%%%%%%%%%%%%%%
% دستورهایی برای سفارشی کردن صفحات اول فصل‌ها
\makeatletter
\newcommand\mycustomraggedright{%
 \if@RTL\raggedleft%
 \else\raggedright%
 \fi}
\def\@part[#1]#2{%
\ifnum \c@secnumdepth >-2\relax
\refstepcounter{part}%
\addcontentsline{toc}{part}{\thepart\hspace{1em}#1}%
\else
\addcontentsline{toc}{part}{#1}%
\fi
\markboth{}{}%
{\centering
\interlinepenalty \@M
\ifnum \c@secnumdepth >-2\relax
 \Huge\bfseries \partname\nobreakspace\thepart
\par
\vskip 20\p@
\fi
\Huge\bfseries #2\par}%
\@endpart}

\def\@makechapterhead#1{%
\vspace*{50pt} 
%\vspace*{-30\p@}%
{\parindent \z@ \mycustomraggedright 
%{\parindent 0pt \mycustomraggedright 
{\centering{
\ifnum \c@secnumdepth >\m@ne
\if@mainmatter

%\huge\bfseries \@chapapp\space {\chapternumber\thechapter}
%\huge\bnazaninout \@chapapp\space {\chapternumber\thechapter}
{\large\bnazaninout \@chapapp{} \thechapter \par}
\par\nobreak
\vskip 20\p@
\fi
% \vskip 20pt 
\fi
\interlinepenalty\@M 
\Huge \bfseries #1\par\nobreak  \vskip 120\p@  }}}}
% \Huge \bf #1       \par \nobreak \vskip 40pt }}}}
\makeatother
%%%%%%%%%%%%%%%%%%%%%%%%%%%%
%دستورات مربوط به نحوه شماره‌دهی جدول ها و شکل‌ها
\makeatletter 
\renewcommand \theequation {\@arabic\c@equation.\@arabic\c@section.\@arabic\c@chapter}
\renewcommand \thefigure     {\@arabic\c@figure.\@arabic\c@section.\@arabic\c@chapter}
\renewcommand \thetable      {\@arabic\c@table.\@arabic\c@section.\@arabic\c@chapter}
% دستوری برای حذف خط در زیر سربرگ هر صفحه
\renewcommand{\headrulewidth}{0pt}
\makeatother
%%%%%%%%%%%%%%%%%%%%%%%%%%%%
\begin{document}
% در این فایل، عنوان پایان‌نامه، مشخصات خود، متن تقدیمی‌، ستایش، سپاس‌گزاری و چکیده پایان‌نامه را به فارسی، وارد کنید.
% توجه داشته باشید که جدول حاوی مشخصات پایان‌نامه/رساله و همچنین، مشخصات داخل آن، به طور خودکار، درج می‌شود.
%%%%%%%%%%%%%%%%%%%%%%%%%%%%%%%%%%%%
% دانشگاه خود را وارد کنید
\university{تربیت مدرس}
% دانشکده، آموزشکده و یا پژوهشکده  خود را وارد کنید
\faculty{علوم ریاضی}
% گروه آموزشی خود را وارد کارشناسی ارشد آمار (رشته)}
\degree {کارشناسی ارشد علوم کامپیوتر}

% گروه آموزشی خود را وارد کنید
\subject{علوم کامپیوتر }
% گرایش خود را وارد کنید
\field{داده کاوی}
% عنوان پایان‌نامه را وارد کنید
\title{روش های عمیق مبتنی برمبدل های بینایی در تحلیل داده های تصویری}
% نام استاد(ان) راهنما را وارد کنید
\firstsupervisor{آقای دکتر منصور رزقی}
%\secondsupervisor{استاد راهنمای دوم}
% نام استاد(دان) مشاور را وارد کنید. چنانچه استاد مشاور ندارید، دستور پایین را غیرفعال کنید.
%\firstadvisor{نام استاد(دان) مشاور}
%نام استاد مشاور را وارد کنید. چنانچه دو استاد مشاور دارید  دستور پایین را نیز فعال کنید.
%\secondadvisor{نام استاد(دان) مشاور}
% نام پژوهشگر را وارد کنید
\name{سید محمد}
% نام خانوادگی پژوهشگر را وارد کنید
\surname{بادزهره}
% تاریخ پایان‌نامه را وارد کنید
\thesisdate{نابستان ۱۴۰۴}
% کلمات کلیدی پایان‌نامه را وارد کنید
%\keywords{داده‌های بقا فضایی، مدل‌های شکنندگی فضایی}
% چکیده پایان‌نامه را وارد کنید
\newpage
%\fa-abstract{\noindent
%در 

%}
%\newpage
\thispagestyle{empty}
\vtitle
%\newpage
%\thispagestyle{empty}
%\clearpage
%~~~
%\newpage
%\thispagestyle{empty}
%\input{rights}
%\newpage
%\thispagestyle{empty}
%\clearpage
%~~~
%\newpage
%\thispagestyle{empty}
%\centerline{{\includegraphics[width=20 cm]{replyrecord}}}
%\newpage
%\thispagestyle{empty}
%\clearpage
%~~~
\newpage
 % پایان‌نامه خود را تقدیم کنید!
\begin{acknowledgementpage}

\vspace{4cm}

{\nastaliq
{\Large
 تقدیم به
\vspace{1.5cm}

\newdimen\xa
\xa=\textwidth
% \advance \xa by -14cm
% \hspace{\xa}
پدر بزرگوار و مادر مهربانم و برادر عزیزم 
\newline
آن ها که از خواسته هایشان گذشتند، سختی ها را به جان خریدند و خود را سپر بلای مشکلات و ناملایمات کردند تا من به جایگاهی که اکنون در آن ایستاده ام برسم .


% \vspace{1.5cm}

\newdimen\xa
\xa=\textwidth
\advance \xa by -12cm
\hspace{\xa}
%...
\vspace{1.5cm}

\newdimen\xa
\xa=\textwidth
\advance \xa by -10cm
\hspace{\xa}

\vspace{1.5cm}

\newdimen\xa
\xa=\textwidth
\advance \xa by -8cm
\hspace{\xa}
}}
\end{acknowledgementpage}
%\newpage
%\thispagestyle{empty}
%\clearpage
%~~~
%%%%%%%%%%%%%%%%%%%%%%%%%%%%%%%%%%%%
%\newpage
%\thispagestyle{empty}
% ستایش
%\baselineskip=.750cm
%\ \\ \\
% 
%{\nastaliq
%رسيدن، به دانش است و به كردار نیک...%
%}\\
%\vspace{.5cm}\\
%{\scriptsize\nastaliq
%{
% و 
% }}
% 
%\vspace{.5cm}
%{\nastaliq
%\newdimen\xb
%\xb=\textwidth
%\advance \xb by -8.5cm
%\hspace{\xb}
%پس منطق ناگزير آمد بر جوينده‌ی رستگاری.
%\RTLfootnote{مقدمه‌ی رساله‌ی منطق دانشنامه‌ی علائی، شیخ‌الرئیس ابن‌سینا}
%}
%\newpage
%\thispagestyle{empty}
%\clearpage
%~~~
%%%%%%%%%%%%%%%%%%%%%%%%%%%%%%%%%%%%
\newpage
\thispagestyle{empty}
%قدردانی
{\nastaliq
قدردانی \newline
 \newline
 
از استاد گرانقدر، جناب آقای دکتر رزقی که با راهنمایی‌های دلسوزانه و ارزشمند خود، همواره در مسیر تحقیق این پایان‌نامه یار و راهنمای من بودند، نهایت سپاس و قدردانی را دارم.
\newline

از خانواده عزیزم که با محبت بی‌پایان، صبوری و حمایت‌های بی‌دریغ‌شان، همواره پشتیبان من در طی این مسیر سخت و پرچالش بودند، صمیمانه سپاسگزارم.

}
{\scriptsize\nastaliq
%سپاس و ستایش خدایی را که هستی نام ازو یافت، فلک جنبش، زمین آرام ازو یافت.
}


% با استفاده از دستور زیر، امضای شما، به طور خودکار، درج می‌شود
\signature 
%
\newpage
\thispagestyle{empty}
\baselineskip=1.0cm
\ \\ \\
\begin{large}
\begin{center}
چکیده
\end{center}
\end{large}
%}}
%\\[4cm]
ببعلعذدقفلهعقفدلخقفدللقفلقفاقا
%\newpage
%\thispagestyle{empty}
%\clearpage
%~~~
%\newpage
%{\small
%\abstractview
%}
%\newpage
%\thispagestyle{empty}
%\clearpage
%~~~
\newpage
\pagenumbering{harfi}
\pagenumbering{alph}
\tableofcontents
\listoftables
\listoffigures
\clearpage{\pagestyle{empty}\cleardoublepage}
\pagenumbering{arabic}
{
\baselineskip=1.0cm
\chapter*{پیش‌گفتار}
\phantomsection
\addcontentsline{toc}{chapter}{پیش‌گفتار}



\section*{پیش‌گفتار}

مبدل‌های بینایی\footnote{\lr{vision transformers}}در سال‌های اخیر توجه زیادی در حوزه بینایی ماشین و یادگیری عمیق به خود جلب کرده‌اند. این مدل‌ها با وجود عملکرد قوی، دارای تعداد زیادی پارامتر هستند و معمولاً ساختار فضایی داده را از بین می‌برند. استفاده از روش‌های تانسوری راه‌حلی برای این مسئله ارائه می‌دهد.

در این پژوهش، چارچوبی مبتنی بر ساختارهای تانسوری برای بهینه‌سازی مبدل‌های بینایی طراحی شده است. هدف از این کار، کاهش تعداد پارامترها، حفظ ساختار داده و بهبود تفسیرپذیری مدل است. از لایه‌های فشرده‌سازی تانسوری (\lr{TCL}) و رگرسیون تانسوری (\lr{TRL}) برای مدل‌سازی نگاشت‌های چندبعدی استفاده شده است.

این پایان‌نامه شامل چهار فصل است:

- \textbf{فصل اول}: بیان مقدمه، مسئله تحقیق، اهداف پژوهش و اهمیت موضوع.
- \textbf{فصل دوم}: مرور پیشینه پژوهش، معرفی ساختار مبدل‌های بینایی و مبدل‌های پنجره متحرک.
- \textbf{فصل سوم}: تشریح روش پیشنهادی، طراحی مدل تانسوری و پیاده‌سازی معماری مبدل پنجره متحرک به صورت تانسوری.
- \textbf{فصل چهارم}: ارزیابی مدل، تحلیل نتایج حاصل از آزمایش‌ها و مقایسه عملکرد مدل پیشنهادی با مدل‌های پایه.

در پایان، جمع‌بندی، نتیجه‌گیری و پیشنهادهایی برای پژوهش‌های آینده ارائه شده است.




\chapter{مفاهیم اولیه}
در این فصل به معرفی مقدمات و مفاهیم مورد نیاز در این پایان‌نامه می‌پردازیم. 
\section{مقدمه}
در این بخش به تاریخچه هوش مصنوعی، دستاورد های اولیه، چالش ها، دلایل رکود هوش مصنوعی و پایان عصر تاریک هوش مصنوعی صحبت میکنیم 

% update the abstract


\subsection{آغاز هوش مصنوعی و هدف اصلی}

هوش مصنوعی به عنوان شاخه ای از علوم کامپیوتر، در دهه 1950 با هدف ساخت  سیستم ها و ماشین هایی که توانایی تقلد از هوش انسانی را دارند، آغاز شد.
نخستین بار مکارتی در سال 1956 این اصطلاخ را به کار گرفت. و هوش مصنوعی به عنوان علمی که در آن به مطالعه الگوریتم هایی برای تقلید رفتار انسانی می پردازد، شناخته شد.
اهداف اولیه هوش مصنوعی شامل توانایی درک زبان، یادگیری، حل مسئله و تولید موجودات هوشمند بود.
در این دوران پروژه های تحقیقاتی زیادی به امید دستیابی به هوش مصنوعی عمومی 
 AGI (Artificial General Intelligence)
شروع به کار کردند

\subsection{دورهٔ طلایی و پیشرفت‌های اولیه}
در دهه 50 و60 میلادی، هوش مصنوعی به عنوان یکی از پرچمداران پژوهش های نوین شناخته می شد.
الگوریتم های اولیه به کمک روش های منطقی و ریاضیاتی برای حل مسئله و بازی های ساده توسعه یافتند مانند انواع الگوریتم جستوجوی درختی که در این دوره به وجود آمدند و زمینه ساز اولین دستاوردهای هوش مصنوعی در بازی های تخته ای همچون شطرنج شدند.
در این دوران پیشرفت های بیشتری در پردازش زبان طبیعی (NLP)
و سیستم های خبره (Expert Systems) نیز صورت گرفت که این امید را در دانشمندان و محققان تقویت کرد که دستیابی به هوش مصنوعی عمومی به زودی ممکن خواهد بود.

\subsection{انتظارات بیش از حد و ظهور عصر تاریک}
با وجود پیشرفت های های هوش مصنوعی، محدودیت های تکنولوژی مثل gpu ها  و محاسباتی در آن زمان و همچنین کمبود داده های کافی برای آموزش مدل های پیچیده تر، باعث شد که بسیاری از پروژه های تحقیقاتی نتوانند به نتایج پیش بینی شده دست یابند. و در نتیجه، هوش مصنوعی در دهه 70 به مرحله ای از رکود وارد شد که به آن عصر تاریک هوش مصنوعی یا (AI Winter) می گویند.
در این دوران بسیاری از پروژه ها تعطیل و سرمایه گذاری ها قطع شدند و دولت ها و سازمان های سرمایه گذار به دلیل عدم دستیابی به نتایج مطلوب از ادامه سرمایه گذاری منصرف شدند.

\subsection{عوامل اصلی عصر تاریک هوش مصنوعی}
\begin{itemize}
	\item \textbf{محدودیت‌های سخت‌افزاری:}
در آن زمان، سیستم‌های اولیه هوش مصنوعی  به محاسبات سنگینی نیاز داشتند که با توان پردازشی محدود آن زمان همخوانی نداشت.
	
	\item \textbf{کمبود داده‌ها:}
 در آن زمان، دسترسی به داده‌های کافی برای آموزش مدل‌های پیچیده ممکن نبود و الگوریتم‌های موجود به داده‌های بیشتری نیاز داشتند تا بتوانند به درستی آموزش ببینند و عملکرد مطلوبی داشته باشند.
	
	\item \textbf{روش‌های محدود یادگیری:} 
الگوریتم‌های اولیه به شدت به برنامه‌ریزی انسانی وابسته بودند و در بسیاری از موارد، مدل‌ها قادر به تعمیم به مسائل جدید نبودند و نمی توانستند تعمیم پذیری خیلی بالایی داشته باشند. \cite{russell2016artificial}.
	\end{itemize}

\subsection{پایان عصر تاریک و بازگشت هوش مصنوعی}
پس از چندین سال رکود و عدم سرمایه گذاری در حوزه هوش مصنوعی، سرانجام در دهه 1980 و 1990  عصر تاریک هوش مصنوعی با تحولات تکنولوژی و از همه مهم تر ظهور سیستم های خبره
(Expert-Systems)
به پایان رسید.
سیستم های خبره به عنوان یکی از اولین تلاش های موفق برای کاربردهای صنعتی در هوش مصنوعی به وجود آمدند.بر خلاف الگوریتم های اولیه، این سیستم هااز پایگاه بزرگ قواعد و قوانین 
(Rule-Based-Systems) 
استفاده میکردند. 
در سیستم های خبره به جای تلاش برای شبیه سازی کلی هوش مصنوعی، بر حل مسائل تخصصی برای صنایع و سازمان ها تمرکز میکردند. برای مثال، سیستم های خبره در پزشکی برای تشخیص بیماری ها و پیشنهاد درمان، در صنعت برای مدیریت و پیش بینی خرابی ماشین آلات، و در امور مالی برای تحلیل و ارزیابی ریسک کاربرد داشتند.
هر چند این سیستم ها نمی توانستند درک عمیق و هوشمندی عمومی را ایجاد کنند. اما برای رفع نیاز های پیچیده مناسب بودند.
همزمان با موفقیت این سیستم ها، بهبودهای زیادی در سخت افزارها و کاهش هزینه های پردازش به وجود آمد. در دهه های 1980  و 1990 کامپیوتر ها به تدریج قوی تر و مقرون به رفه تر شدند و امکان پردازش داده های بیشتر و اجرای الگوریتم های پیچیده تر فراهم شد.
این افزایش توان محاسباتی، نیاز به پردازش داده های بزرگ و پیچیده را فراهم کرد و در نتیجه دسترسی به داده ها و انجام محاسبات سنگین برای توسعه الگوریتم های جدید فراهم شد.
از طرف دیگر، پیشرفت های انجام شده در ذخیره سازی داده و رشد اینترنت باعث دسترسی گسترده تر به داده ها و منابع اطلاعاتی شد.
به این ترتیب، مجموعه ای از عوامل شامل ظهور سیستم های خبره، افزایش قدرت پردازش و دسترسی به داده های بیشتر، منجر به بازگشت هوش مصنوعی شد و این دوره نه تنها پایان عصر هوش مصنوعی بود، بلکه راه را برای الگوریتم های یادگیری ماشین و توسعه شبکه های عصبی هموار کرد. \cite{(McCorduck, 2004; Russell & Norvig, 2016).}


\section{انواع مدل یادگیری ماشین و شبکه های عصبی}

\subsection{۱.۲.۱. یادگیری ماشین: مروری کلی}

یادگیری ماشین (Machine-Learning) شاخه‌ای از هوش مصنوعی است که به مدل های محاسباتی این امکان را می دهد الگو ها  از داده ها را به طور خودکار یاد بگیرند و بتوانند تصمیم گیری کنند در واقع، هدف یادگیری ماشین این است که مدل ها بتوانند  از داده ها الگو ها و روابط پنهان را استخراج کنند و به نتایج و رفتار و تصمیم های قابل اعتماد دست یابند.
\subsection{تقسیم‌بندی‌های اصلی در یادگیری ماشین}

یادگیری ماشین به سه دستهٔ اصلی تقسیم می‌شود:
یادگیری با نظارت(Supervised-Learning)
یادگیری بدون نظارت (Unsupervised-Learning)
یادگیری تقویتی (Reinforcement-Learning)

\subsection{یادگیری نظارت شده (Supervised-Learning)}
یادگیری نظارت شده یکی از رایج ترین روش ها در یادگیری ماشین شناخته می شود. که در آن از مجموعه داده های برچسب گذاری شده برای آموزش مدل استفاده می کنیم.
هدف این الگوریتم تشخیص الگو ها در میان داده های ورودی است که این امکان را می دهد  پیش بینی یا طبقه بندی هایی روی داده جدید انجام دهد.
این نوع شامل دو الگوریتم Regression  و classification  می شود.

\subsubsection{طبقه‌بندی (Classification)}



\chapter{پیشینه پژوهش}

\section{مقدمه}
ظهور مبدل‌ها یکی از تحولات اساسی در حوزهٔ پردازش زبان طبیعی \footnote{\lr{NLP}} و یادگیری ماشین به شمار می‌رود.

\cite{vaswani2017attention,bahdanau2014neural}. این مدل‌ها باعث تغییرات عمده‌ای در نحوهٔ ساخت و آموزش مدل‌های زبانی و همچنین در بسیاری از کاربردهای دیگر یادگیری ماشین شده‌اند و توانستند بسیاری از مشکلات مدل‌های قبلی را حل کنند\cite{devlin2018bert,radford2018improving}.
\section{مشکلات ترجمه ماشینی و مبدل ها}

در ابتدا، ترجمه ماشینی \footnote{\lr{machine translation}} یک چالش اساسی در زمینهٔ پردازش زبان طبیعی بود. مدل‌های اولیه‌ای مانند مدل‌های مبتنی بر قواعد \footnote{\lr{Rule-based Models}} برای ترجمه استفاده می‌شدند که در آن‌ها، ترجمه‌ها به‌صورت دستی با استفاده از قواعد زبانی مشخص تنظیم می‌شدند \cite{nagao1984framework,hutchins1986machine}. این روش‌ها هرچند دقیق بودند، اما محدودیت‌های زیادی داشتند و نمی‌توانستند ویژگی‌های پیچیده‌تر زبان را مدل‌سازی کنند.



سپس مدل‌های آماری \footnote{\lr{Statistical Models}} معرفی شدند \cite{brown1993mathematics,koehn2010statistical}. این مدل‌ها از داده‌های ترجمه‌شده برای آموزش مدل‌های آماری استفاده می‌کردند که احتمال ترجمهٔ صحیح را براساس شواهد آماری محاسبه می‌کردند. مدل‌هایی مانند مدل‌های ترجمهٔ آماری مبتنی بر جمله \footnote{\lr{Phrase-based Statistical Models}} \cite{koehn2003statistical} از این نوع بودند که قادر به ترجمهٔ جملات بهتر از مدل‌های مبتنی بر قواعد بودند، اما هنوز هم در ترجمه‌های پیچیده با مشکلاتی روبه‌رو بودند.



بعد از این مدل‌ها، مدل‌های بازگشتی \footnote{\lr{Recurrent Models}} به وجود آمدند که مشکلات آن‌ها در فصل گذشته بیان شد \cite{elman1990finding,sutskever2014sequence}. در نهایت، این مشکلات باعث به وجود آمدن ترانسفورمرها شد \cite{bahdanau2014neural}.




\section{ظهور ترانسفورمرها}
در سال ۲۰۱۷، مقاله‌ای توسط گوگل \footnote{\lr{google}} منتشر شد که مفهوم جدیدی به نام مبدل ها \footnote{\lr{Transformers}} را معرفی کرد \cite{vaswani2017attention}. این مقاله به موضوع ترجمهٔ ماشینی پرداخت و نشان داد که با استفاده از مکانیزم توجه می‌توان بسیاری از مشکلات مدل‌های قبلی را حل کرد \cite{luong2015effective}.

مدل‌های ترانسفورمر برخلاف مدل‌های قبلی که از پردازش سریالی استفاده می‌کردند، از پردازش موازی بهره می‌برند. این ویژگی به ترانسفورمرها اجازه می‌دهد که به‌طور همزمان به تمام بخش‌های ورودی توجه کنند. این قابلیت باعث شد که ترانسفورمرها در پردازش تصویر و متن بسیار سریع‌تر و دقیق‌تر از مدل‌های قبلی عمل کنند \cite{vaswani2017attention}.

\section{معماری ترانسفورمرها}
در \textbf{تصویر \ref{fig:transformer_architecture}}، معماری ترانسفورمر نمایش داده شده است و بخش‌ها و اجزای مختلف آن مشخص شده است. معماری ترانسفورمر از دو بخش اصلی تشکیل شده است:
\begin{itemize}
\item \textbf{رمزگذار \footnote{\lr{Encoder}}:}

	وظیفهٔ انکودر این است که دادهٔ ورودی را دریافت کند و ویژگی‌های آن را استخراج کند.
\item \textbf{رمزگشا \footnote{\lr{Decoder}}:}

	وظیفهٔ دیکودر این است که ویژگی‌های استخراج‌شده را به زبان مقصد تبدیل کند.
\end{itemize}

\begin{figure}[h]
	\centering
	\includegraphics[width=0.65\textwidth]{Transformer-model-architecture.png}
	\caption{معماری ترانسفورمرها}
	\label{fig:transformer_architecture}
\end{figure}

\subsection{جاسازی}
ددر زبان طبیعی، کلمات به شکل رشته‌های متنی هستند مانند کتاب، ماشین و ... کامپیوترها نمی‌توانند به‌طور مستقیم این کلمات را به شکل رشته‌های متنی پردازش کنند. به همین دلیل، در یادگیری ماشین این کلمات را به شکل یک بردار نمایش می‌دهیم. این بردار بیانگر آن کلمه در مدل است تا ماشین بتواند آن کلمه را پردازش کند \cite{bengio2003neural}.

این بردارها ویژگی‌های کلمه را در فضای عددی نمایش می‌دهند. روش‌های مختلفی برای تبدیل متن به بردار وجود دارند. از جمله این روش‌ها می‌توان به روش‌های \lr{Word2Vec} \cite{mikolov2013distributed} و \lr{GloVe} \cite{pennington2014glove} اشاره کرد.

همان‌طور که در \autoref{fig:word_embedding} نشان داده شده است، هر کلمه که به صورت توکن است، ابتدا در دیکشنری تعریف‌شده پیدا می‌شود و پس از پیدا شدن در دیکشنری، با استفاده از روش‌های \footnote{\lr{Embedding}}، هر کلمه به برداری از اعداد تبدیل می‌شود. این جاسازی‌ها شباهت‌های معنایی بین کلمات را مدل‌سازی می‌کنند و کلماتی که از نظر معنایی شبیه به هم هستند، بردار آن‌ها نیز به یکدیگر نزدیک‌تر است. به این ترتیب، کلمات برای مدل‌ها و شبکه‌های عصبی قابل‌فهم می‌شوند \cite{mikolov2013distributed,pennington2014glove}.





 \begin{figure}[h]
 	\centering
 	\begin{minipage}[b]{0.7\textwidth}
 		\centering
 		\includegraphics[width=\textwidth]{transformer_images/word_embedding.png}
 		\caption{\lr{word embedding}}
 		\label{fig:word_embedding}
 	\end{minipage}
 	\hfill
 \end{figure}



\subsection{جاسازی موقعیتی}

تا الان هر کلمه را به برداری از اعداد که برای مدل قابل فهم باشد، تبدیل کرده‌ایم. اما مدل‌های ترانسفورمر نمی‌توانند جایگاه هر کلمه را تشخیص دهند. در مدل‌های ترانسفورمر، برخلاف مدل‌های بازگشتی، به دلیل اینکه کلمات به‌صورت موازی وارد می‌شوند، نیاز داریم تا جایگاه هر کلمه را بدانیم. به‌طور مثال، در جملهٔ «من تو را دوست دارم» باید به‌طور دقیق بدانیم که «من» کلمهٔ اول جمله است، «تو» کلمهٔ دوم جمله است و ... .

حال باید به مدل توالی این کلمات را بفهمانیم. بنابراین، نیاز داریم به مدل یک سری اطلاعات اضافی بدهیم به‌طوری‌که مدل توالی کلمات را یاد بگیرد. روش‌های مختلفی برای اضافه‌کردن جاسازی موقعیتی \footnote{\lr{Positional Embedding}} به مدل وجود دارد. در ترانسفورمرها از روش جاسازی موقعیت سینوسی
 \footnote{\lr{Sinusoidal Positional Embedding}} استفاده می‌شود \cite{vaswani2017attention}.



این روش قابل یادگیری نیست و صرفاً از یک سری فرمول‌های ساده برای جاسازی موقعیتی استفاده می‌کند.  
برای موقعیت \footnote{\lr{pos}} در توالی و بُعد \( i \) در فضای برداری، تعبیهٔ موقعیتی به‌صورت زیر تعریف می‌شود:  


\begin{equation}
	PE(pos, 2i) = \sin\left( \frac{pos}{10000^{\frac{2i}{d}}} \right)
	\label{eq:pe_even}
\end{equation}

و برای مقادیر فرد:  

\begin{equation}
	PE(pos, 2i+1) = \cos\left( \frac{pos}{10000^{\frac{2i}{d}}} \right)
	\label{eq:pe_odd}
\end{equation}
	



\begin{itemize}
	\item \( pos \): موقعیت کلمه در توالی است (مثلاً از \( 0 \) تا \( N-1 \) برای یک توالی \( N \)-تایی).
	\item \( i \): شاخص بعد در بردار موقعیتی (از \( 0 \) تا \( d-1 \) برای بعد فضای برداری \( d \)).
	\item \( d \): ابعاد فضای برداری مدل که نشان می‌دهد هر کلمه در چند بعد نمایش داده می‌شود.
	\item \( 10000 \): یک مقدار ثابت برای تنظیم مقیاس توابع تناوبی و ایجاد فرکانس‌های مختلف در ابعاد گوناگون.
\end{itemize}  

همان‌طور که در شکل \autoref{fig:word_embedding_positional_embedding} مشاهده می‌کنید، بعد از جاسازی کلمات، به آن جا سازی موقعیتی اضافه می‌شود. در این روش از توابع سینوس و کسینوس استفاده می‌شود. این توابع موقعیت‌ها را در فضای برداری به‌گونه‌ای نگاشت می‌کنند که مدل بتواند از ترتیب کلمات در توالی آگاه باشد \cite{vaswani2017attention}. این ویژگی به مدل کمک می‌کند تا توالی زمانی را درک کرده و الگوهای زمانی را شبیه‌سازی کند. از مزایای این روش می‌توان به عدم نیاز به آموزش و توزیع متوازن جایگاه کلمات اشاره کرد.

\begin{figure}[h]
	\centering
	\begin{minipage}[b]{0.7\textwidth}
		\centering
		\includegraphics[width=\textwidth]{transformer_images/positional_embedding.png}
		\caption{\lr{word embedding + positional embedding}}
		\label{fig:word_embedding_positional_embedding}
	\end{minipage}
	\hfill
\end{figure}






\subsection{توجه}



در روش‌ شبکه های بازگشتی، توالی ورودی (مثلاً یک جمله) معمولاً به‌صورت گام‌به‌گام پردازش می‌شد \cite{elman1990finding,hochreiter1997long}. اما در ترانسفورمر می‌خواهیم مدلی داشته باشیم که به هر موقعیت (مثلاً یک کلمه) در توالی نگاه کند و به همهٔ موقعیت‌های دیگر نیز به‌صورت موازی دسترسی داشته باشد. به این مفهوم \textbf{توجه} می‌گوییم.

به زبان ساده، وقتی توکن (کلمه) \( i \) به توکن‌های دیگر نگاه می‌کند، می‌خواهد بداند کدام توکن‌ها برای تفسیر معنای خودش مهم‌ترند.

به طور مثال در جمله‌ی «یک گربه روی زمین نشسته است» می‌خواهد بداند کلمه‌ی «گربه» به واژه‌ی «نشستن» بیشتر توجه کند یا به «زمین». در این‌جا فعل «نشستن» ارتباط نزدیک‌تری به «گربه» دارد و از نظر معنایی مرتبط‌تر است.



\[
Q = \text{Query (پرسش)}, \quad K = \text{Key (کلید)}, \quad V = \text{Value (مقدار / ارزش)}
\]





در  ضرب شباهت های توجه \footnote{\lr{Scaled Dot-Product Attention}}    \cite{vaswani2017attention}، ابتدا شباهت یا ارتباط بین پرسش \footnote{\lr{Query}} و کلید \footnote{\lr{Key}} را با محاسبهٔ ضرب داخلی \footnote{\lr{Dot Product}} به‌دست می‌آوریم، سپس آن را نرمال می‌کنیم (با تقسیم بر \( d_k \)) و از تابع سافت مکس \footnote{\lr{softmax}} استفاده می‌کنیم تا ضرایب توجه \footnote{\lr{Attention Weights}} را به‌دست آوریم. در نهایت با همین ضرایب، ترکیبی خطی از بردارهای مقدار \footnote{\lr{value}} را می‌گیریم.

فرمول به‌شکل زیر است:

\begin{equation}
	\text{Attention}(Q, K, V) = \text{softmax}\left( \frac{QK^T}{\sqrt{d_k}} \right) V
	\label{eq:attention}
\end{equation}

که در آن:

\[
Q \in \mathbb{R}^{n \times d_k} \quad \text{ماتریس پرسش برای }
\]
\[
K \in \mathbb{R}^{n \times d_k} \quad \text{ماتریس کلید برای }
\]
\[
V \in \mathbb{R}^{n \times d_v} \quad \text{ماتریس مقدار  }
\]

تقسیم بر \( d_k \) باعث می‌شود مقدار ضرب داخلی در ابعاد بالا خیلی بزرگ نشود و شیب‌ها گرادیان پایدار بمانند.

\begin{equation}
	\alpha = \text{softmax}\left( \frac{QK^T}{\sqrt{d_k}} \right)
	\label{eq:alpha}
\end{equation}
\(\alpha\) یک ماتریس با ابعاد \( n \times n \) است که سطر \( i \)-ام آن ضرایب توجه برای توکن \( i \) را نشان می‌دهد.

تفسیر ضرایب توجه: هر سطر از \( \alpha \) نشان می‌دهد که توکن فعلی به چه توکن‌هایی در جمله، با چه شدتی توجه می‌کند.



\begin{figure}[h]
	\centering
	\begin{minipage}[b]{0.7\textwidth}
		\centering
		\includegraphics[width=\textwidth]{transformer_images/multi_head_attention_new.png}
		\caption{توجه}
		\label{fig:attention}
	\end{minipage}
	\hfill
	
\end{figure}





ایدهٔ چندسری \footnote{\lr{multi head attention}}  
به‌جای آنکه فقط یک‌بار \( Q, K, V \) بسازیم و عملیات توجه را انجام دهیم، چندین مجموعهٔ متفاوت \( Q_i, K_i, V_i \) می‌سازیم (هر کدام یک «سر» \footnote{\lr{head}} یا سر نام دارد) و به‌صورت موازی محاسبات توجه را انجام می‌دهیم. سپس خروجی همهٔ این سرها را کنار هم قرار داده \footnote{\lr{concatenate}} و در نهایت با یک ماتریس وزن دیگر ضرب می‌کنیم تا به بعد اصلی بازگردیم.

فرمول مربوط به این ایده به‌شکل زیر است:


\begin{equation}
	\text{head}_i = \text{Attention}(Q_i, K_i, V_i)
	\label{eq:head_i}
\end{equation}


\begin{equation}
	\text{MultiHead}(Q, K, V) = [\text{head}_1 \oplus \cdots \oplus \text{head}_h] W_O
	\label{eq:multihead}
\end{equation}

که در آن \( \oplus \) نشان‌دهندهٔ عمل الحاق \footnote{\lr{concatenate}} است.

ماتریس وزن \( W_O \) به‌شکل زیر است:

\[
W_O \in \mathbb{R}^{(h \cdot d_v) \times d_{\text{model}}}
\]

که \( W_O \) ماتریسی است که خروجی الحاق‌شده را به بعد \( d_{\text{model}} \) برمی‌گرداند.





\begin{figure}[h]
	\centering
	\begin{minipage}[b]{0.9\textwidth}
		\centering
		\includegraphics[width=\textwidth]{transformer_images/multi_head_attention.png}
		\caption{\lr{multi head attention}}
		\label{fig:attention}
	\end{minipage}
	\hfill
	
\end{figure}



\subsection*{چرا چندین سر؟}

\textbf{مشاهدهٔ چند منظر متفاوت:} هر  سر می‌تواند الگوهای گوناگونی از وابستگی‌ها را بیاموزد (مثلاً یک سر می‌تواند یاد بگیرد کلمهٔ فعلی با کلمات همسایهٔ نزدیک خود بیشتر مرتبط شود، یک سر  دیگر روی ارتباط با کلماتی در فاصلهٔ دورتری متمرکز باشد، سر دیگر روی مطابقت جنس و تعداد در دستور زبان و ...).

\textbf{افزایش ظرفیت مدل:} با داشتن چند سر ، مدل می‌تواند قدرت بیان بیشتری داشته باشد.

\textbf{ابعاد کمتر در هر سر:} در عمل، اگر \( d_{\text{model}} \) مثلاً 512 باشد، و تعداد سر ها \( h = 8 \)، آنگاه هر سر  ابعادی در حدود \( d_k = 64 \) خواهد داشت؛ و این محاسبات ضرب داخلی را نیز مقیاس‌ پذیر و قابل موازی‌سازی می‌کند.


\subsection{اتصال باقی مانده}
در معماری‌های عمیق، هنگامی که تعداد لایه‌ها زیاد می‌شود، اغلب دچار ناپایداری گرادیان می‌شوند و این مشکل باعث دشواری در آموزش مدل می‌گردد \cite{hochreiter1997long,bengio1994learning}. 

در مبدل ها \cite{vaswani2017attention}، به جای این که خروجی توجه را به‌صورت مستقیم به لایهٔ بعدی بدهیم، ورودی آن را نیز حفظ کرده و به خروجی اضافه می‌کنیم. ایدهٔ اصلی این روش از اتصالات باقی‌مانده \footnote{\lr{Residual Connection}} در شبکه‌های عمیق الهام گرفته شده است \cite{he2016deep}.

اگر \( x \) ورودی به زیرماژول و \( \text{SubLayer}(x) \) خروجی آن زیرماژول باشد، در انتهای کار عبارت زیر را محاسبه می‌کنیم:

\begin{equation}
	x + \text{SubLayer}(x)
	\label{eq:sublayer}
\end{equation}

این جمع به‌صورت عنصر به‌عنصر \lr{\lr{Element-wise Addition}} انجام می‌شود.

\subsection{مزایای اتصال باقی مانده در ترانسفورمر}

\subsection*{کمک به جریان یافتن گرادیان}
وقتی ورودی مستقیماً به خروجی اضافه می‌شود، مسیری مستقیم برای عبور شیب (گرادیان) به عقب ایجاد می‌گردد.
در صورت نبود این اتصال، اگر شبکه عمیق شود، گرادیان‌ها ممکن است در لایه‌های پایین محو شوند و عملاً \lr{gradient vanishing} رخ دهد \cite{hochreiter1997long,bengio1994learning}.

\subsection*{حفظ اطلاعات اصلی (هویت ورودی)}
حتی اگر زیرماژول تغییری در اطلاعات ورودی ایجاد کند، با وجود \lr{Residual Connection}، ورودی اصلی همواره در خروجی نهایی حضور دارد.
این ویژگی باعث می‌شود در صورت ناکافی‌بودن یادگیری زیرماژول یا در مراحل اولیهٔ آموزش، دست‌کم بخشی از سیگنال (اطلاعات) خام به لایه‌های بالاتر برسد \cite{he2016deep,vaswani2017attention}.

\subsection*{کاهش ریسک تخریب ویژگی‌ها}
در شبکه‌های عمیق، یکی از مشکلات این است که هر لایه ممکن است بخشی از اطلاعات مفید را تخریب کند. اتصال باقی مانده تضمین می‌کند که اگر لایه‌ای به هر دلیل نتوانست الگوی بهینه را یاد بگیرد، اطلاعات قبلی حداقل به‌صورت دست‌نخورده تا حدی منتقل می‌شود.


\section{نرمال سازی لایه ها}

در یادگیری عمیق، نرمال‌سازی \footnote{\lr{Normalization}} داده‌های یک لایه یا فعال‌سازی‌ها، اغلب به سرعت بخشیدن به همگرایی و پایدار کردن آموزش کمک شایانی می‌کند. شاید معروف‌ترین نوع نرمال‌سازی، نرمال سازی بج \footnote{\lr{Batch Normalization}} باشد که پیش‌تر در کارهای بینایی کانولوشنی\footnote{\lr{CNN}} بسیار مورداستفاده قرار گرفت \cite{ioffe2015batch}.

نرمال سازی لایه ها \footnote{\lr{Layer Normalization}} روشی جایگزین است که در ترانسفورمر استفاده می‌شود \cite{ba2016layer,vaswani2017attention}. علت اصلی این انتخاب، ماهیت توالی‌محور \footnote{\lr{sequence}} بودن داده‌ها در پردازش زبان طبیعی و عدم تمایل به وابستگی به آمار مینی‌بچ است.

\subsection*{تفاوت نرمال سازی بچ ها با نرمال سازی لایه ها }

\subsection*{\lr{Batch Normalization}}
در نرمال سازی بچ ها، برای نرمال‌سازی، میانگین و واریانس روی تمام نمونه‌های موجود در مینی‌بچ \footnote{\lr{mini-batch}} (و نیز در طول ابعاد ویژگی) محاسبه می‌شود \cite{ioffe2015batch}.
این موضوع در پردازش زبان طبیعی کمی دردسرساز است؛ چون ترتیب توکن‌ها، طول جمله‌ها و حتی اندازهٔ مینی‌بچ ممکن است نامنظم باشد.
همچنین به خاطر تنوع طول توالی‌ها، پیاده‌سازی نرمال سازی بچ ها می‌تواند پیچیده شود.

\subsection*{\lr{Layer Normalization}:}
در  نرمال سازی لایه ها، برای هر توکن به‌صورت جداگانه (در طول بُعد ویژگی)، میانگین \footnote{\lr{mean}} و واریانس \footnote{\lr{variance}} گرفته می‌شود \cite{ba2016layer}.
فرض کنید در یک لایه، بردار \( h_i \in \mathbb{R}^{d_{\text{model}}} \) مربوط به توکن \( i \) باشد؛ یعنی ابعاد ویژگی آن \( d_{\text{model}} \) است. ما میانگین \( \mu_i \) و واریانس \( \sigma_i^2 \) را از اجزای این بردار محاسبه می‌کنیم:

\begin{equation}
	\mu_i = \frac{1}{d_{\text{model}}} \sum_{k=1}^{d_{\text{model}}} h_{i,k}, \quad
	\sigma_i^2 = \frac{1}{d_{\text{model}}} \sum_{k=1}^{d_{\text{model}}} (h_{i,k} - \mu_i)^2
	\label{eq:mean_variance}
\end{equation}


سپس نرمال‌سازی برای هر مؤلفهٔ \( k \) در بردار توکن \( i \) به شکل زیر انجام می‌شود:

\begin{equation}
	\hat{h}_{i,k} = \frac{h_{i,k} - \mu_i}{\sqrt{\sigma_i^2 + \epsilon}}
	\label{eq:normalized_h}
\end{equation}

در نهایت، برای این‌که مدل بتواند مقیاس و بایاس جدیدی یاد بگیرد، شبیه بچ نرم ، دو پارامتر \( \gamma \) و \( \beta \) نیز در طول بعد ویژگی اعمال می‌شوند:

\begin{equation}
	\text{LayerNorm}(h_i) = \gamma \odot \hat{h}_i + \beta
	\label{eq:layernorm}
\end{equation}


که در آن \( \gamma, \beta \in \mathbb{R}^{d_{\text{model}}} \) هستند و \( \odot \) ضربِ عنصر به عنصر است \cite{ba2016layer}.

\subsection*{مزایای نرمال سازی لایه در مبدل ها}

\begin{itemize}
	\item \textbf{بی‌نیازی از وابستگی به ابعاد مینی‌بچ:}  
	با نرمال سازی لایه، می‌توان حتی با اندازهٔ مینی‌بچ برابر ۱ نیز به‌خوبی آموزش دید، چراکه آمارها وابسته به ابعاد ویژگی‌اند و نه مینی‌بچ \cite{ba2016layer}.
	
	\item \textbf{پایدارسازی توزیع فعال‌سازی‌ها:}  
	زمانی که مدل در حال یادگیری است، توزیع‌های داخلی لایه‌های میانی ممکن است تغییر کند.\footnote{\lr{Internal Covariate Shift}} نرمال سازی لایه با نرمال‌سازی این توزیع، آموزش را پایدارتر و سریع‌تر می‌کند \cite{ioffe2015batch,ba2016layer}.
	
	\item \textbf{سازگاری با داده‌های توالی‌محور:}  
	هر توکن را جداگانه نرمال می‌کند و نگرانی‌ای بابت ترتیب طول جمله‌ها، یا قرار گرفتن چند جملهٔ کوتاه/بلند در یک مینی‌بچ نداریم \cite{vaswani2017attention}.
\end{itemize}

در معماری مبدل ها، پس از خروجی هر زیرماژول، مراحل به‌شکل زیر است:

\paragraph{اتصال باقی مانده}
ابتدا ورودی همان زیرماژول (مثلاً بردار \( x \)) را با خروجی زیرماژول (\( \text{SubLayer}(x) \)) جمع می‌کنیم. حاصل این جمع را می‌توان چنین نوشت:


\begin{equation}
	z = x + \text{SubLayer}(x)
\end{equation}


این \( z \) حالا ترکیبی از اطلاعات اصلی ورودی و اطلاعات یادگرفته‌شده توسط \lr{SubLayer} است.

\paragraph{نرمال سازی لایه}
سپس این بردار \( z \) را وارد لایهٔ \lr{LayerNorm} می‌کنیم:

\[
y = \text{LayerNorm}(z)
\]

خروجی نهایی را می‌توان به لایهٔ بعدی پاس داد یا به مرحلهٔ بعدی در همین لایه.

به‌عبارتی اگر بخواهیم در یک فرمول واحد بیان کنیم:

\[
\text{Add \& Norm} = \text{LayerNorm}\bigl(x + \text{SubLayer}(x)\bigr)
\]


\section{رمزگشا}
دیکودر در معماری ترانسفورمرها وظیفهٔ تولید خروجی نهایی را بر عهده دارد. این خروجی معمولاً می‌تواند توالی هدف \footnote{\lr{Target Sequence}} باشد، مانند ترجمهٔ یک جمله یا پیش‌بینی توکن‌های بعدی در یک توالی \cite{vaswani2017attention}. 

در این بخش، دیکودر دو ورودی اصلی دارد:  
1. توالی هدف که معمولاً به‌صورت خودکار تولید می‌شود (مثلاً در ترجمهٔ ماشینی یا تولید متن)،  
2. نمایش اطلاعات کدشده که توسط انکودر تولید شده است و شامل ویژگی‌های استخراج‌شده از توالی ورودی می‌باشد.  

دیکودر از این ورودی‌ها استفاده می‌کند تا به‌صورت گام‌به‌گام، خروجی نهایی خود را تولید کند \cite{bahdanau2014neural,sutskever2014sequence}.

همان‌طور که در شکل \ref{fig:Decoder} مشاهده می‌کنید، دیکودر دو ورودی دارد.

\begin{figure}[h]
	\centering
	\begin{minipage}[b]{0.25\textwidth}
		\centering
		\includegraphics[width=\textwidth]{transformer_images/decoder.png}
		\caption{Decoder}
		\label{fig:Decoder}
	\end{minipage}
	\hfill
\end{figure}

تمامی بخش‌های دیکودر مانند انکودر هستند اما در دیکودر توجه چند سر ماسک شده \footnote{\lr{Masked Multi-Head Attention}} وجود دارد \cite{vaswani2017attention}.



\section{توجه  چند سری ماسک شده}
در مبدل‌ها، مکانیزم توجه چند سری \footnote{\lr{Multi-Head Attention}} در بخش دیکودر به‌صورت ماسک شده \footnote{\lr{Masked}} پیاده‌سازی می‌شود تا مدل نتواند توکن‌های آینده را ببیند و به‌صورت خودبازگشتی \footnote{\lr{Autoregressive}} توکن بعدی را پیش‌بینی کند \cite{vaswani2017attention}.


در واقع ایدهٔ اصلی استفاده از ماسک جلوگیری از مشاهدهٔ آینده است.

در معماری‌های خودبازگشتی، مدل در گام \( i \) از دیکودر تنها باید به توکن‌های قبلی \( \{ y_1, \dots, y_{i-1} \} \) دسترسی داشته باشد؛ اما نه به توکن‌های \( \{ y_{i+1}, y_{i+2}, \dots \} \). اگر مدل بتواند توکن‌های آینده را «نگاه» کند، پیش‌بینی توکن بعدی آسان و غیرواقعی می‌شود (مشکل نشت اطلاعات) \cite{bahdanau2014neural,sutskever2014sequence}.

به همین دلیل در توجه چند سری ماسک شده در  دیکودر، از یک ماتریس ماسک \( M \) استفاده می‌کنیم که اجازه نمی‌دهد هر توکن به توکن‌های آینده‌اش توجه کند.

\section{مثال عددی توجه ماسک شده}
فرض کنید دنبالهٔ 4 توکنی داریم:
\[
[y_1, y_2, y_3, y_4]
\]
خروجی ضرب داخلی  (قبل از \texttt{softmax}) یک ماتریس \(4 \times 4\) خواهد بود:
\[
S =
\begin{bmatrix}
	s_{1,1} & s_{1,2} & s_{1,3} & s_{1,4} \\
	s_{2,1} & s_{2,2} & s_{2,3} & s_{2,4} \\
	s_{3,1} & s_{3,2} & s_{3,3} & s_{3,4} \\
	s_{4,1} & s_{4,2} & s_{4,3} & s_{4,4}
\end{bmatrix}
\]

\begin{itemize}
	\item \textbf{سطر 1 (توکن اول)}: تنها می‌تواند خودش (\textbf{ستون 1}) را ببیند، اما \textbf{ستون‌های 2 تا 4} ماسک می‌شوند.
	\item \textbf{سطر 2 (توکن دوم)}: می‌تواند به \textbf{ستون‌های 1 و 2} نگاه کند، اما \textbf{ستون‌های 3 و 4} ماسک می‌شوند.
	\item \textbf{سطر 3}: می‌تواند \textbf{ستون‌های 1، 2 و 3} را ببیند، اما \textbf{ستون 4} ماسک می‌شود.
	\item \textbf{سطر 4}: می‌تواند به \textbf{ستون‌های 1، 2، 3 و 4} دسترسی داشته باشد (چهارمین توکن می‌تواند توکن‌های قبلی را ببیند. همچنین این توکن \textbf{خودش} نیز معمولاً در دسترس است — بسته به پیاده‌سازی، ممکن است توکن فعلی از خودش نیز استفاده کند یا نه. در معماری استاندارد، سطر \( i \) معمولاً به ستون \( i \) هم دسترسی دارد).
\end{itemize}

در عمل، ماتریس ماسک \( M \) به شکل زیر خواهد بود (با نشانه‌گذاری پایین‌مثلثی):


\[
M =
\begin{bmatrix}
	0 & -\infty & -\infty & -\infty \\
	0 & 0 & -\infty & -\infty \\
	0 & 0 & 0 & -\infty \\
	0 & 0 & 0 & 0
\end{bmatrix}
\]

به این ترتیب، پس از جمع شدن با \( S \) و اجرای \texttt{softmax} در هر سطر، ضرایب توجه ستون‌های ماسک‌شده به صفر میل می‌کنند \cite{vaswani2017attention}.

 
\section{مبدل های بینایی}

ایدهٔ ترانسفورمرها در حوزهٔ بینایی \footnote{\lr{vision transformer}} از تعمیم ترانسفورمر متن به تصاویر به وجود آمده است \cite{dosovitskiy2020image}.

ما در این بخش از مبدل های بینایی برای وظیفه کلاس‌بندی استفاده می‌کنیم.

در روش‌های متداول برای پردازش تصویر، از کانولشن \footnote{\lr{convolution}} ‌های متوالی استفاده می‌کردند؛ اما در ترانسفورمرها تصاویر به پچ‌های مختلف شکسته می‌شوند \cite{dosovitskiy2020image}. هر پچِ شکسته‌شده از تصویر می‌تواند با سایر پچ‌ها به‌صورت موازی وارد مکانیزم توجه  شود و شباهت یا ارتباطشان با یکدیگر سنجیده شود. در بخش‌های بعد، به‌طور مفصل روند انجام این کار را توضیح خواهیم داد.


\subsection{جاسازی پچ ها در مبدل های بینایی}
در ترانسفورمرهای مبتنی بر متن، هر کلمه به توکن تبدیل می‌شود و سپس هر کلمه به برداری تبدیل می‌گردد. این بردارها پس از افزودن جاسازی موقعیتی وارد مکانیزم توجه می‌شوند \cite{vaswani2017attention}. 

حال همین ایده در تصویر پیاده‌سازی شده است. همان‌طور که در شکل \ref{fig:image to patch in vision transformer} مشاهده می‌کنید، در مبدل های بینایی، به‌جای استفاده از عملیات کانولوشن‌های متوالی که در شبکه‌های CNN مرسوم است \cite{lecun1998gradient,krizhevsky2012imagenet,he2016deep}، تصویر را به بلاک‌های غیرهم‌پوشان (\(P \times P\)) تقسیم می‌کنیم. این کار علاوه بر ساده‌سازی موازی‌سازی، به مدل اجازه می‌دهد از سازوکار \lr{Self-Attention} برای ارتباط بین این بلاک‌ها استفاده کند \cite{dosovitskiy2020image}.

\begin{figure}[h]
	\centering
	\begin{minipage}[b]{0.9\textwidth}
		\centering
		\includegraphics[width=\textwidth]{transformer_images/image_patch_embedding.png}
		\caption{image to patch}
		\label{fig:image to patch in vision transformer}
	\end{minipage}
	\hfill
\end{figure}

\subsection{شکل پچ‌ها:}
فرض کنید ابعاد تصویر ورودی (\(H \times W \times C\)) باشد. به‌عنوان مثال، اگر اندازهٔ تصویر \(224 \times 224 \times 3\) باشد، طول و عرض تصویر به‌ترتیب \(224\) و تصویر دارای سه کانال رنگی است:

\[
H = 224, \quad W = 224, \quad C = 3
\]

\begin{figure}[h]
	\centering
	\begin{minipage}[b]{0.9\textwidth}
		\centering
		\includegraphics[width=\textwidth]{transformer_images/space_original_image.png}
		\caption{original Image}
		\label{fig:Original Image}
	\end{minipage}
	\hfill
\end{figure}

حال اگر اندازهٔ هر پچ (\(P \times P\)) باشد (برای نمونه \(16 \times 16\))، تصویر به‌صورت یک جدول مشبک از پچ‌های کوچک تقسیم می‌شود.  
به هر پچ می‌توان مانند یک «کاشی» از تصویر نگاه کرد:
- پچ اول: مختصات \((0 \text{ تا } 15 \text{ در ارتفاع}) \) و \((0 \text{ تا } 15 \text{ در عرض})\)،  
- پچ دوم: مختصات \((0 \text{ تا } 15 \text{ در ارتفاع}) \) و \((16 \text{ تا } 31 \text{ در عرض})\)،  
- و به همین ترتیب تا کل تصویر پوشش داده شود.

\begin{figure}[h]
	\centering
	\begin{minipage}[b]{0.9\textwidth}
		\centering
		\includegraphics[width=\textwidth]{transformer_images/space_patch_iamge.png}
		\caption{\lr{paches of image}}
		\label{fig:Patch Image}
	\end{minipage}
	\hfill
\end{figure}

\subsection{تعداد پچ‌ها:}
اگر پچ‌ها بدون هم‌پوشانی باشند، ابعاد پچ باید بر ابعاد تصویر بخش‌پذیر باشد.  

- تعداد پچ‌های افقی: \(\frac{W}{P}\)  
- تعداد پچ‌های عمودی: \(\frac{H}{P}\)  

در مجموع:  
\begin{equation}
	\left(\frac{H}{P}\right) \times \left(\frac{W}{P}\right) = \frac{H}{P} \times \frac{W}{P}.
	\label{eq:grid_size}
\end{equation}


برای مثال اگر:
\[
H = 224, \quad W = 224, \quad P = 16:
\]
\[
\frac{224}{16} = 14 \quad \Rightarrow \quad 14 \times 14 = 196 \quad (\text{تعداد پچ‌ها}).
\]

در اکثر نسخه‌های مبدل‌های بینایی، پچ‌ها بدون هم‌پوشانی \footnote{\lr{Non-overlapping}} هستند. اندازهٔ پچ‌های کوچک باعث می‌شود تعداد پچ‌ها زیاد شود و در نتیجه هزینهٔ توجه بالا رود. از طرفی، پچ‌های بزرگ هزینهٔ توجه را کاهش می‌دهند؛ اما ممکن است جزییات محلی \footnote{\lr{Local Details}} را از دست بدهیم \cite{dosovitskiy2020image}.


\subsection{بردارکردن هر پچ}
هر پچ دارای ابعاد \((P \times P \times C)\) است. برای مثال اگر \(P=16\) و \(C=3\)، آنگاه پچ ابعاد \(16 \times 16 \times 3\) خواهد داشت.  
برای این‌که بتوانیم پچ‌ها را مانند «توکن»‌های پردازش زبان ظبیعی به مبدل ها بدهیم، باید آن‌ها را به یک بردار یک ‌بعدی تبدیل کنیم. در صورت قرار دادن پیکسل‌های پچ به‌صورت ردیفی \footnote{\lr{Row-major}}، طول این بردار خواهد بود:

\begin{equation}
	P \times P \times C = P^2 \times C.
	\label{eq:patch_volume}
\end{equation}

در مثال \((16 \times 16 \times 3)\)، طول بردار می‌شود \(768\).


\section{اعمال لایهٔ خطی}
بعد از کنار هم چیدن پچ ها \footnote{\lr{Flatten}} کردن، معمولاً یک لایهٔ خطی \footnote{\lr{Fully-Connected Layer}} روی این بردار اعمال می‌شود تا آن را به بعد \(d_{\text{model}}\) (مثلاً 768 یا 1024) ببرد. در حقیقت، این لایه یک تبدیل ویژگی \footnote{\lr{Feature Transformation}} انجام می‌دهد تا همهٔ پچ‌ها یک نمایندگی (\lr{Embedding}) با ابعاد یکنواخت \(d_{\text{model}}\) پیدا کنند:
\[
(P^2 \times C) \quad \rightarrow \quad d_{\text{model}}
\]

\begin{figure}[h]
	\centering
	\begin{minipage}[b]{0.9\textwidth}
		\centering
		\includegraphics[width=\textwidth]{transformer_images/vision_transformer_embedding.png}
		\caption{مبدل های بینایی}
		\label{fig:Embedding Vision Transformer}
	\end{minipage}
	\hfill
\end{figure}

این مرحله شبیه ساخت توکن در پردازش زبان طبیعی است؛ با این تفاوت که در پردازش زبان طبیعی، توکن «کلمه» یا «زیرکلمه» است و از قبل دارای بردار تعبیه‌شده جاساز شده بوده است \cite{vaswani2017attention}. در مبدل های بینایی \cite{dosovitskiy2020image}، ما ابتدا باید تصاویر را پچ کنیم و سپس بردارهای  جاساز را از این پچ‌ها به دست آوریم.

ترانسفورمر نیاز دارد ورودی‌اش توالی توکن‌ها باشد. در پردازش زبان طبیعی توالی کلمات داریم، در مبدل های بینایی توالی «پچ»‌ها:
\[
\{ x_{\text{patch}_1}, x_{\text{patch}_2}, \dots, x_{\text{patch}_N} \}.
\]
هر پچ اکنون یک بردار \(d_{\text{model}}\)-بعدی است. پس یک مجموعه با طول \(N\) (تعداد پچ‌ها) و عرض \(d_{\text{model}}\) خواهیم داشت.  
اگر عدد پچ‌ها \(N\) باشد (مثلاً 196)، ترانسفورمر می‌تواند با مکانیزم توجه خود سر، وابستگی  میان پچ‌ها را یاد بگیرد: کدام بخش از تصویر برای کدام بخش دیگر مهم‌تر است، چگونه ترکیب جهانی \footnote{\lr{Global Context}} شکل گیرد.
\cite{vaswani2017attention,dosovitskiy2020image}.

معمولاً پچ‌ها را به‌صورت ردیفی شماره‌گذاری می‌کنند (ابتدا پچ‌های ردیف بالایی از چپ به راست، سپس ردیف بعدی و …)، تا مدل در صورت نیاز بتواند از موقعیت‌ها، اطلاعات مکانی تقریبی داشته باشد.  
در عمل، چون قصد داریم (در مراحل بعد) به هر پچ یک جاسازی موقعیتی هم اضافه کنیم، مکان دقیق هر پچ در بُعد دوم (ویژگی) کد می‌شود.

در مبدل بینایی \cite{dosovitskiy2020image} دیگر به کانولوشن وابسته نیستیم. در عوض، از جاساازی استفاده می‌شود.  
تقسیم  کردن تصویر به بلاک‌های \((P \times P)\)، کنار هم چیدن و تبدیل آن به جاساز همگی عملیات ریاضی ساده‌ای هستند که به‌راحتی روی \lr{GPU}/\lr{TPU} قابل موازی‌سازی‌اند.

\subsection{توکن کلاس بندی}
توکن کلاس بندی \footnote{\lr{Cls Token}} یک بردار ویژه است که به ابتدای دنبالهٔ ورودی اضافه می‌شود و نقش آن، خلاصه‌کردن اطلاعات کل ورودی (چه متن، چه تصویر) است \cite{devlin2018bert,dosovitskiy2020image}.

در مبدل بینایی، این توکن در ابتدای پچ‌های تصویری قرار می‌گیرد.  
این توکن یک بردار با ابعاد \(d_{\text{model}}\) است (همان ابعاد سایر توکن‌ها) و پارامتری یادگرفتنی محسوب می‌شود؛ یعنی مدل طی آموزش، مقادیر آن را برای ذخیره و تجمیع اطلاعات بهینه می‌کند.

در وظایف دسته‌بندی کلاس بندی، هدف این است که یک پیش‌بینی کلی برای کل ورودی (مثلاً یک جمله یا یک تصویر) ارائه دهیم؛ توکن کلاس بندی دقیقاً همین وظیفه را بر عهده دارد \cite{devlin2018bert}. این توکن از طریق مکانیزم توجه چند سر در مبدل ها با تمامی توکن‌های دیگر (پچ‌های تصویر) ارتباط می‌گیرد و اطلاعات مهم آن‌ها را در لایه‌های مختلف مبدل ها را به‌صورت تجمعی یاد می‌گیرد. به عبارتی، توکن کلاس بندی نقش نمایندهٔ کل تصویر یا متن را بر عهده دارد.

توکن کلاس بندی از طریق ضرب داخلی در مکانیزم توجه، می‌تواند به تمام پچ‌ها نگاه کند و با ضرایب توجه (\(\alpha\)) مشخص کند که از هر پچ چه مقدار اطلاعات بگیرد. بدین‌ترتیب، به‌طور ضمنی یاد می‌گیرد روی ویژگی‌هایی که برای دسته‌بندی مهم هستند (نظیر الگوها، اشکال و بخش‌های کلیدی تصویر) متمرکز شود.

در طول لایه‌های ترانسفورمر، توکن کلاس بندی نقش محوری در خلاصه‌سازی بازنمایی کل تصویر ایفا می‌کند. این توکن به‌صورت پارامتر قابل یادگیری تعریف شده و در طول فرآیند آموزش به‌روزرسانی می‌شود \cite{devlin2018bert,dosovitskiy2020image}.

\subsection{انکودر در مبدل های بینایی}

انکودر در ترانسفورمرها همانند مبدل اصلی است \cite{vaswani2017attention}، با این تفاوت که در مبدل های بینایی \cite{dosovitskiy2020image} دیگر به دیکودر نمی‌رویم. پس از عبور از بلاک‌های ترانسفورمر، در ساده‌ترین حالت یک لایهٔ خطی (\lr{Fully Connected}) یا یک لایهٔ \lr{MLP (Multi-Layer Perceptron)} بر روی بردار نهایی اعمال می‌شود و این لایه‌ها به تعداد کلاس‌ها خروجی می‌دهند.  
سپس خروجی هر لایه با گذر از تابع سافت مکس به احتمال هر کلاس تبدیل می‌شود و در نهایت مدل کلاس با بیشترین احتمال را به‌عنوان خروجی پیش‌بینی می‌کند.

\begin{figure}[h]
	\centering
	\begin{minipage}[b]{0.9\textwidth}
		\centering
		\includegraphics[width=\textwidth]{transformer_images/vision_transformer_after_embedding.png}
		\caption{توکن توجه در مبدل های بینایی}
		\label{fig:Cls Token In Vision Transformer}
	\end{minipage}
	\hfill
\end{figure}

در مبدل ها، هر لایهٔ انکودر و دیکودر با پردازش عمیق‌تر روی توالی ورودی، می‌تواند نمایش بهتری از ویژگی‌ها به‌دست بیاورد \cite{vaswani2017attention}. تکرار چندین‌بارهٔ انکودر یا دیکدر موجب می‌شود مدل بتواند ساختارهای پیچیده‌ای را یاد بگیرد و کیفیت و دقت آن در شناسایی توالی‌های طولانی و معانی پنهان افزایش یابد \cite{vaswani2017attention,dosovitskiy2020image}. در نتیجه، مدل با تعداد لایه‌های بیشتر اغلب عملکرد بهتری از خود نشان می‌دهد.

   
   
\section{مبدل پنجره‌ای متحرک}
ایدهٔ مبدل پنجره‌ای متحرک \footnote{\lr{Swin Transformer}} از ترکیب چند مفهوم کلیدی در مدل‌های ترانسفورمر و شبکه‌های کانولوشنی شکل گرفت \cite{vaswani2017attention,he2016deep,liu2021swintransformer}.

یکی از بزرگترین مشکلات در ترانسفورمرهای اولیه، نیاز به محاسبات بسیار زیاد در زمانی بود که تصویر ورودی ابعاد بسیار بزرگی داشت \cite{dosovitskiy2020image}. در ترانسفورمر معمولی هر پچ به تمامی پچ‌های دیگر توجه  می‌کرد و در مواقعی که تعداد پچ‌ها زیاد می‌شد، هزینهٔ محاسباتی و حافظه به‌شدت افزایش پیدا می‌کرد.

در شبکه‌های کانولوشنی، معماری معمولاً به‌صورت سلسله‌مراتبی پیش می‌رود \cite{he2016deep}؛ یعنی ابتدا ویژگی‌های محلی استخراج می‌شود، سپس با عمیق‌تر شدن لایه‌ها، این ویژگی‌ها در سطوح بالاتر با یکدیگر ترکیب می‌شوند. در مبدل پنجره‌ای متحرک \cite{liu2021swintransformer}، با دانش بر این موضوع توانسته‌اند هم هزینه‌های محاسباتی را کاهش دهند و هم دقت مدل را افزایش دهند.

\begin{figure}[h]
	\centering
	\begin{minipage}[b]{1\textwidth}
		\centering
		\includegraphics[width=\textwidth]{transformer_images/swin_transformer.png}
		\caption{مبدل پنجره متحرک}
		\label{fig: swin transformer}
	\end{minipage}
	\hfill
\end{figure}

در مبدل پنجره‌ای متحرک، به‌جای آن‌که مدل به تمام پچ‌ها در یک سطح ویژگی نگاه کند، تصویر را به «پنجره‌های محلی» \footnote{\lr{Local Windows}} تقسیم می‌کند و توجه را محدود به همان ناحیه می‌سازد \cite{liu2021swintransformer}. سپس با تکنیک جابه‌جایی \footnote{\lr{Shift}} این پنجره‌ها در لایه‌های بعدی، توان مدل برای ترکیب اطلاعات از نواحی مختلف تصویر (و در نهایت دیدن کل تصویر) افزایش پیدا می‌کند. این رویکرد، ایدهٔ کلیدی‌ای بود که باعث شد مدل هم محاسبات سبک‌تری داشته باشد و هم بتواند ارتباط‌های جهانی \footnote{\lr{Global}} را در طول لایه‌ها به‌دست آورد.



یکی دیگر از ایده‌های مهم در در مبدل پنجره‌ای متحرک، کوچک‌ کردن تدریجی نقشهٔ ویژگی  در طول معماری است؛ مشابه کاری که در \lr{ResNet} یا سایر \lr{CNN}ها انجام می‌شود \cite{he2016deep}. این امر ضمن کاهش هزینهٔ محاسباتی، باعث می‌شود مدل بتواند با سطوح مختلفی از ویژگی‌ها کار کند و در نهایت خروجی نهایی باکیفیت‌تری ارائه دهد.

\subsection{قطعه‌بندی پچ}
فرض کنیم تصویر ورودی \(\displaystyle I\) دارای ابعاد \(\displaystyle (H \times W \times 3)\) باشد. گام نخست، تقسیم تصویر به پچ‌های کوچک \(\displaystyle (P \times P)\) است \cite{dosovitskiy2020image}. اگر \(P\) اندازهٔ پچ (\lr{Patch size}) باشد، آنگاه تعداد پچ‌ها در بعد افقی و عمودی، به‌ترتیب \(\displaystyle \frac{H}{P}\) و \(\displaystyle \frac{W}{P}\) خواهد بود. هر پچ را می‌توان به‌صورت یک بردار درآورد:

\[
X_{\text{patch}} \in \mathbb{R}^{(P^2 \cdot 3)}.
\]

سپس کل تصویر به \(\displaystyle \frac{H}{P} \times \frac{W}{P}\) پچ تبدیل خواهد شد و در نتیجه، ماتریس \(\displaystyle X\) از کنار هم قرار گرفتن این پچ‌ها به صورت زیر به‌دست می‌آید:

\[
X \in \mathbb{R}^{\Bigl(\frac{H}{P} \cdot \frac{W}{P}\Bigr) \times \Bigl(P^2 \cdot 3\Bigr)}.
\]

\subsection{جاسازی}
در ادامه، برای این‌که بتوانیم هر پچ را در یک فضای برداری با بعد \(\displaystyle C\) (ابعاد مدل) نمایش دهیم، یک لایهٔ خطی (\lr{Fully Connected Layer}) روی هر پچ اعمال می‌شود \cite{dosovitskiy2020image,liu2021swintransformer}:

\begin{equation}
	Z = X \cdot W_{\text{embed}} + b_{\text{embed}}, 
	\quad
	Z \in \mathbb{R}^{\Bigl(\tfrac{H}{P} \cdot \tfrac{W}{P}\Bigr) \times C}.
\end{equation}

در عمل، این عملیات معادل یک تبدیل خطی ساده است:

\[
W_{\text{embed}} \in \mathbb{R}^{\bigl(P^2 \cdot 3\bigr) \times C},
\quad
b_{\text{embed}} \in \mathbb{R}^{C}.
\]

پس از این مرحله، ما در هر موقعیت \((h, w)\) (از شبکهٔ پچ‌ها) یک بردار 
\(\displaystyle z_{h,w} \in \mathbb{R}^{C}\) داریم. این ماتریس \(\displaystyle Z\) 
ورودیِ اولین مرحله (\lr{Stage}) از مبدل های پنجره متحرک خواهد بود \cite{liu2021swintransformer}.

هر بلوک مبدل پنجره متحرک از چند بخش اصلی تشکیل شده است \cite{liu2021swintransformer}:

\begin{itemize}
	\item پنجره‌بندی تصویر \footnote{\lr{Window Partition}} 
	یا پنجره‌بندی جابه‌جاشده \footnote{\lr{Shifted Window Partition}}
	\item اعمال توجه چمد سر پنجره ای \footnote{\lr{Window Multi-Head Self Attention}}
	\item لایهٔ \lr{Skip Connection}\footnote{\lr{Skip Connection}} و \lr{Layer Norm}\footnote{\lr{Layer Norm}}
	\item مسیر پرسیپترون چندلایه \footnote{\lr{MLP}}:
	\begin{itemize}
		\item یک لایهٔ \lr{MLP} شامل دو لایهٔ \lr{Fully-Connected}\footnote{\lr{Fully-Connected}} 
		و تابع فعال‌ساز \lr{GeLU}\footnote{\lr{GeLU}} (یا تابع مشابه)
		\item لایهٔ \lr{Skip Connection}\footnote{\lr{Skip Connection}} و \lr{Layer Norm}\footnote{\lr{Layer Norm}}
	\end{itemize}
\end{itemize}




\subsection{توجه چند سر پنجره ای}

\subsubsection{تعریف پنجره‌های محلی}

در مبدل های پنجره متحرک، به‌جای آن‌که تمام پیکسل‌های یک نقشهٔ ویژگی بزرگ را یک‌جا 
در محاسبهٔ توجه  درگیر کنیم، نقشهٔ ویژگی را به قطعه‌های کوچکی به‌اندازه
\(\displaystyle (M \times M)\) تقسیم می‌کنیم. این قطعه‌های کوچک را 
«پنجره‌های محلی» می‌نامیم.

اگر اندازهٔ نقشهٔ ویژگی در یک لایه 
\(\displaystyle (H' \times W')\) باشد، 
با تقسیم آن به پنجره‌های 
\(\displaystyle (M \times M)\)، 
در راستای طول تقریباً 
\(\displaystyle \tfrac{H'}{M}\) پنجره خواهیم داشت 
و در راستای عرض هم 
\(\displaystyle \tfrac{W'}{M}\) پنجره.
(برای راحتی، فرض می‌کنیم 
\(\displaystyle H'\) و \(\displaystyle W'\) 
دقیقاً مضربی از \(\displaystyle M\) باشند 
تا تقسیم بدون باقی‌مانده انجام شود.)

هر کدام از این پنجره‌های 
\(\displaystyle (M \times M)\) 
دارای 
\(\displaystyle M^2\) پیکسل (یا موقعیت مکانی) است، 
و در هر پیکسل هم یک بردار ویژگی با بعد \(\displaystyle C\) قرار دارد.

به بیان ساده‌تر:
\begin{itemize}
	\item نقشهٔ ویژگی مثل یک صفحهٔ بزرگ است.
	\item آن را مانند شطرنج به مربع‌های کوچکی \(\displaystyle (M \times M)\) بخش می‌کنیم.
	\item در هر مربع (پنجره)، فقط به همان مربع نگاه می‌کنیم و محاسبات  توجه  را انجام می‌دهیم.
	\item این کار باعث می‌شود تعداد پیکسل‌هایی که درگیر محاسبهٔ توجه هستند، 
	به‌مراتب کمتر شود و هزینهٔ محاسباتی کاهش یابد.
\end{itemize}


\subsection{توجه}

برای هر بلوک، ابتدا بردارهای پرسش، کلید، مقدار ساخته می‌شوند. 
اگر \(\displaystyle z_i \in \mathbb{R}^C\) بردار ورودی مربوط به موقعیت \(i\) باشد، آنگاه:

\[
q_i = z_i W_Q, 
\quad
k_i = z_i W_K,
\quad
v_i = z_i W_V,
\]
که 
\[
W_Q, W_K, W_V \;\;\in \;\;\mathbb{R}^{C \times d}.
\]

پارامتر \(\displaystyle d\) معمولاً به‌صورت \(\displaystyle \tfrac{C}{h}\) در نظر گرفته می‌شود 
که در آن \(\displaystyle h\) تعداد سربندی  سر ها است. 
در توجه چند سر، خروجی نهایی با ترکیب \(\displaystyle h\) سر توجه محاسبه می‌شود.

در یک سر توجه، توجه به‌صورت زیر تعریف می‌شود:

\[
\mathrm{Attention}(Q, K, V)
=
\mathrm{Softmax}\Bigl(\frac{QK^\top}{\sqrt{d}}\Bigr)\,V,
\]

که در آن:

\begin{itemize}
	\item \(\displaystyle Q, K, V\) به‌ترتیب ماتریس‌هایی هستند که از کنار هم قرار دادن 
	\(\displaystyle q_i, k_i, v_i\) (برای تمام پیکسل‌های آن پنجره) ساخته می‌شوند.
	\item \(\displaystyle \sqrt{d}\): عامل مقیاس‌کننده برای جلوگیری از بزرگ شدن بیش‌ازحد ضرب داخلی است.
\end{itemize}

در مبدل های پنجره متحرک، این محاسبات به‌صورت پنجره‌ای انجام می‌شوند؛ یعنی 
برای هر پنجره، تنها پیکسل‌های داخل همان پنجره در ماتریس‌های 
\(\displaystyle Q\)، \(\displaystyle K\) و \(\displaystyle V\) لحاظ می‌شوند. 
به این ترتیب، زمان محاسبه و مصرف حافظه به‌شدت کاهش می‌یابد 
(در مقایسه با مبدل های بینایی که همه‌چیز را با هم مقایسه می‌کند).

تعداد سربندی \(\displaystyle h\) معمولاً طوری انتخاب می‌شود که 
\(\displaystyle C = h \times d.\) 
خروجی هر سر پس از محاسبهٔ \lr{Attention} به‌صورت زیر با هم ادغام می‌شوند:

\[
\mathrm{MultiHead}(Q,K,V) 
= 
\bigl[\text{head}_1,\ \text{head}_2,\ \dots,\ \text{head}_h\bigr]\,
W_O,
\]
که 
\[
\text{head}_j = \mathrm{Attention}\bigl(Q_j,\ K_j,\ V_j\bigr),
\quad 
W_O \in \mathbb{R}^{C \times C}
\]
ماتریس ترکیب نهایی است.

\subsection{پنجره متحرک}

در مدبل های پنجر متحرک، ایدهٔ «پنجره‌های جابه‌جاشده»    \footnote{\lr{Shifted Windows}}
به این منظور ارائه شده است تا مدل، ارتباط پیکسل‌های واقع در پنجره‌های مجاور را هم یاد بگیرد \cite{liu2021swintransformer}.
اگر فقط از پنجره‌های ثابت (بدون جابه‌جایی) استفاده کنیم، هر بلوک از تصویر تنها با پیکسل‌های همان 
پنجره در ارتباط خواهد بود و ممکن است اطلاعات نواحی مرزی با نواحی مجاور به‌خوبی تبادل نشود.

روش مبدل های پنجره متحرک برای رفع این محدودیت از یک تکنیک ساده اما مؤثر استفاده می‌کند \cite{liu2021swintransformer}:
\begin{itemize}
	\item در یک لایه، محاسبات توجه در پنجره‌های محلی ثابت انجام می‌شود.
	\item در لایهٔ بعدی، پنجره‌ها به اندازه‌ای مشخص جابه‌جا می‌شوند (به‌صورت شیفت افقی و عمودی) 
	تا نواحی مرزی نیز در محاسبات گنجانده شوند.
	\item این فرآیند باعث می‌شود که پیکسل‌ها در پنجره‌های مختلف (و در مرزهای مختلف) 
	در محاسبات دخیل شوند و تبادل اطلاعات بهتری میان نواحی تصویر رخ دهد.
\end{itemize}

\subsubsection{توجه چند سری پنجره ای}
در توجه چندسری پنجره ای \footnote{\lr{W-MSA}}، نقشهٔ ویژگی به پنجره‌های \(\displaystyle (M \times M)\) تقسیم می‌شود \cite{liu2021swintransformer}.
هیچ جابه‌جایی در این تقسیم‌بندی وجود ندارد؛ یعنی اگر نقشهٔ ویژگی را یک مستطیل بزرگ در نظر بگیریم،
آن را شبیه کاشی‌کاری یا شطرنج‌بندی به بلوک‌های مربعی \(\displaystyle (M \times M)\) برش می‌زنیم.
در این حالت، پیکسل‌های هر پنجره فقط با همدیگر (درون همان پنجره) ارتباط برقرار می‌کنند.

\subsubsection{توجه چند سری پنجره ای جا به جا شده}
مطابق شکل \ref{fig:window vs Shifted Window in Swin Transformer}، بعد از اینکه بلوک اول (توجه چند سری پنجره ای) کارش تمام شد، در بلوک دوم، قبل از تقسیم‌بندی به پنجره‌های 
\(\displaystyle (M \times M)\)، نقشهٔ ویژگی را جابه‌جا  می‌کنیم \cite{liu2021swintransformer}.
در مقالهٔ اصلی، این مقدار جابه‌جایی معمولاً نیمِ اندازهٔ پنجره 
\(\displaystyle \frac{M}{2}\)
در راستای افقی و عمودی است. به این ترتیب:

\begin{itemize}
	\item پیکسل‌هایی که پیش از این در دو پنجرهٔ جداگانه قرار داشتند، ممکن است حالا به دلیل جابه‌جایی وارد یک پنجرهٔ مشترک شوند.
	\item مدل حالا می‌تواند بین این پیکسل‌های «مرزی» نیز  توجه برقرار کند و اطلاعات را بهتر مبادله کند.
\end{itemize}

با این جابه‌جایی، بخشی از پیکسل‌ها در نقشهٔ ویژگی از یک طرف «خارج» می‌شوند. برای اینکه این پیکسل‌ها را از دست ندهیم، از ترفندی به نام  جابجایی چرخه‌ای \footnote{\lr{Cyclic Shift}} استفاده می‌شود. در جا به جایی چرخه ای ، پیکسل‌هایی که از سمت راست بیرون می‌روند دوباره از سمت چپ وارد می‌شوند و بالعکس؛ درست شبیه وقتی که یک تصویر را به‌صورت حلقه‌ای اسکرول می‌کنیم \footnote{\lr{Wrap around}}. مثالی از جا به جایی چرخه ای  در شکل \ref{fig:Cycle Shift in Swin Tranformer} آمده است.

\begin{figure}[h]
	\centering
	\begin{minipage}[b]{1\textwidth}
		\centering
		\includegraphics[width=\textwidth]{transformer_images/cycle_shift.png}
		\caption{جا به جایی چرخه ای}
		\label{fig:Cycle Shift in Swin Tranformer}
	\end{minipage}
	\hfill
\end{figure}

در بلوک اول (بدون جابه‌جایی)، پنجره‌ها ثابت‌اند و پیکسل‌های مرزی در هر پنجره ممکن است فرصت کافی برای تبادل اطلاعات با پیکسل‌های مرزیِ پنجرهٔ کناری را نداشته باشند.

در بلوک دوم (جابه‌جاشده)، مرزهای پنجره‌ها تغییر می‌کند و برخی پیکسل‌هایی که قبلاً در پنجره‌های جدا بودند، اکنون در یک پنجرهٔ مشترک‌اند؛ در نتیجه مدل می‌تواند رابطه و همبستگی بین آن‌ها را هم یاد بگیرد.

این جابه‌جایی و قرارگیری مجدد پیکسل‌ها کنار هم در نهایت کمک می‌کند تا مدل بتواند اطلاعات کل تصویر را با هزینهٔ محاسباتی کمتر (نسبت به توجهِ سراسریِ کامل) در اختیار داشته باشد \cite{liu2021swintransformer}.

اگر بخواهیم با مثال توضیح دهیم، فرض کنید در یک تابلوی شطرنجی، خانه‌های کناری همدیگر را «نمی‌بینند» چون در دو بلوک مختلف هستند.
اما اگر کمی تابلوی شطرنجی را به سمت بالا-چپ یا پایین-راست جابه‌جا کنیم،
حالا بخشی از آن خانه‌ها وارد یک بلوک واحد می‌شوند و اطلاعاتشان با هم ترکیب می‌شود.
سپس به‌طور دوره‌ای (\textit{\lr{Cyclic}})، گوشه‌های اضافی را به آن سمت دیگر تابلوی شطرنجی می‌آوریم
تا هیچ چیز از دست نرود.

به این شکل، سِری اول و دوم بلوک‌های مبدل های پنجره متحرک 
تکمیل‌کنندهٔ یکدیگر می‌شوند \cite{liu2021swintransformer}:
\begin{itemize}
	\item \textbf{بلوک اول:} محاسبهٔ توجه در چهارچوب پنجره‌های ثابت.
	\item \textbf{بلوک دوم:} محاسبهٔ  توجه در پنجره‌های جابه‌جاشده که منجر به تعامل بیشتر بین مرزهای مختلف می‌شود.
\end{itemize}


\subsection{پرسپتروون چند لایه}
پس از انجام توجه چند سری پنجره ای جا  به جا شده 
خروجی به یک مسیر \textbf{\lr{MLP}} می‌رود \cite{liu2021swintransformer}. ساختار این \lr{MLP} به‌صورت زیر است:

\begin{equation}
	X' = \mathrm{GELU}(X W_1 + b_1) \; W_2 + b_2,
	\label{eq:gelu_transform}
\end{equation}

که در آن
\[
W_1 \in \mathbb{R}^{C \times (rC)}, 
\quad
W_2 \in \mathbb{R}^{(rC) \times C}
\]
هستند و \(\displaystyle r\) معمولاً ضریب افزایش بعد را نشان می‌دهد (مثلاً ۴). 

تابع فعال‌ساز \lr{GELU} (یا \lr{ReLU} و سایر توابع) نیز در این‌جا قابل استفاده است \cite{hendrycks2016gelu}.

\subsection{ترکیب پچ ها}
در مدل مبدل های پنجره متحرک، ساختار سلسله‌مراتبی به این معناست که ما در چند مرحله (\lr{Stage}) مختلف، نقشهٔ ویژگی را کوچک‌تر می‌کنیم و در عین حال، عمق (تعداد کانال‌های ویژگی) را افزایش می‌دهیم. هدف اصلی از این کار عبارت است از:

\begin{itemize}
	\item \textbf{استخراج ویژگی‌های سطح بالاتر:}
	وقتی نقشهٔ ویژگی کوچک‌تر می‌شود، هر واحد از نقشهٔ ویژگی بیانگر بخش گسترده‌تری از تصویر اصلی است؛ 
	پس مدل به‌تدریج جزئیات محلی را با درک کلی‌تری از تصویر جایگزین می‌کند \cite{he2016deep}.
	
	\item \textbf{کاهش هزینهٔ محاسبات:}
	در مراحل بعدی، چون ابعاد فضایی کمتر می‌شود، مدل راحت‌تر می‌تواند با ویژگی‌های جدید کار کند 
	(چون مثلاً به‌جای \((H \times W)\) پیکسل، تعداد کمتری پیکسل داریم) \cite{liu2021swintransformer}.
\end{itemize}

این فرایند کوچک‌سازی در این مبدل با نام \footnote{\lr{Patch Merging}} شناخته می‌شود که شبیه به \textit{} در شبکه‌های کانولوشنی ادغام یا پیچش با گام \footnote{\lr{stride convolution}} عمل می‌کند.

\cite{liu2021swintransformer}.

پس از چندین بلوک پردازشی، نقشهٔ ویژگی، ابعادی به شکل \((\tfrac{H}{P}, \tfrac{W}{P})\) با تعداد کانال \(\displaystyle C\) دارد. 
این یعنی پس از برش‌دادن تصویر به پچ‌ها و گذر از چند لایه، اکنون یک نقشهٔ ویژگی داریم که کوچک‌تر از تصویر اصلی است، 
اما هنوز ممکن است خیلی بزرگ باشد.

در مرحلهٔ بعد (\lr{Stage} بعدی)، می‌خواهیم این نقشه را نصف کنیم 
(یعنی طول و عرض را دو برابر کوچک کنیم) و در عوض عمق کانال را دو برابر کنیم 
(تا ظرفیت مدل در استخراج ویژگی‌های پیچیده‌تر بیشتر شود). برای انجام این کار از فرایندی به نام 
ترکیب پچ‌ها استفاده می‌کنیم.
\cite{liu2021swintransformer}:

\subsubsection{1. انتخاب بلوک‌های \((2 \times 2)\)}
ابتدا نقشهٔ ویژگی را در بُعد مکانی به بلوک‌های \((2 \times 2)\) تقسیم می‌کنیم.  
اگر \(\displaystyle Z_{i,j}\) ویژگیِ مکان \((i, j)\) باشد، 
یک بلوک \((2 \times 2)\) شامل چهار پیکسل است:
\[
Z_{2i, 2j}, \quad Z_{2i, 2j+1}, \quad Z_{2i+1, 2j}, \quad Z_{2i+1, 2j+1}.
\]

\subsubsection{2. ادغام ویژگی‌های چهار پیکسل}
برای هر بلوک \((2 \times 2)\)، این چهار پیکسل را در بُعد کانال به هم می‌چسبانیم.  
اگر هر پیکسل یک بردار از بعد \(\displaystyle C\) باشد، اکنون بعدِ حاصل از کنار هم گذاشتن این چهار پیکسل می‌شود \(\displaystyle 4C\).  
نام این بردار ادغام‌شده را \(\displaystyle Z'\) می‌گذاریم.

\subsubsection{3. لایهٔ خطی برای تغییر بعد}
وقتی چهار بردار \(\displaystyle C\)-بعدی را کنار هم می‌گذاریم، یک بردار \(\displaystyle 4C\)-بعدی شکل می‌گیرد.  
حال با یک لایهٔ خطی، بعدِ \(\displaystyle 4C\) را به بعد جدیدی تبدیل می‌کنیم.  
معمولاً این بعد جدید برابر \(\displaystyle 2C\) در نظر گرفته می‌شود؛ 
یعنی دو برابر بزرگ‌تر از قبل اما نه چهار برابر:
\begin{equation}
	Z' \mapsto Z'' = Z' \, W_{\text{merge}} + b_{\text{merge}},
	\label{eq:merge_transform}
\end{equation}

که بعد ویژگی را از \(\displaystyle 4C\) به \(\displaystyle 2C\) کاهش می‌دهد.

\subsubsection{4. کاهش ابعاد مکانی}
در عین حال، وقتی هر چهار پیکسل \((2 \times 2)\) را ادغام می‌کنیم، 
نقشهٔ ویژگی ما ابعاد فضایی \(\bigl(\tfrac{H}{2P} \times \tfrac{W}{2P}\bigr)\) خواهد داشت 
(چون هر بلوک \((2 \times 2)\) تبدیل به یک بردار می‌شود).

به عبارت دیگر، تعداد نقاط مکانی نصف می‌شود (هم در طول و هم در عرض)، 
اما کانال از \(\displaystyle C\) به \(\displaystyle 2C\) افزایش می‌یابد.

\begin{figure}[h]
	\centering
	\begin{minipage}[b]{1\textwidth}
		\centering
		\includegraphics[width=\textwidth]{transformer_images/Patch_merging_new.png}
		\caption{ادغام پچ ها}
		\label{fig:patch merging in Swin Transformer}
	\end{minipage}
	\hfill
\end{figure}

در شبکه‌های کانولوشنی، مرتبا از لایه‌های ادغام \footnote{\lr{Pooling}} یا کانولوشن با گام \footnote{\lr{Stride-Convolution}} 
برای کوچک‌کردن ابعاد استفاده می‌شود تا اطلاعات سطح بالاتر (مثل ساختار کلی اشیا) راحت‌تر استخراج شود \cite{he2016deep}.  در مبدل پنجره متحرک هم همین ایدهٔ سلسله‌مراتب را به دنیای مبدل ها آورده است \cite{liu2021swintransformer}.  
همچنین اگر ابعاد فضایی را کم نکنیم، هزینهٔ توجه به‌شدت زیاد می‌شود 
(چون باید در هر لایه برای همهٔ پیکسل‌ها توجه محاسبه گردد).

در معماری کلی کبدل های پنجره متحرک، پس از \lr{Stage 1} و عبور از بلوک‌های توجه چند سر پنجره ای و توجه چند سر پنجره ای جابه جا شده، عملیات  ادغام پچ ها انجام می‌شود. سپس در \lr{Stage 2}، ویژگی‌های کوچک‌تری داریم، اما تعداد کانال‌ها افزایش یافته است \cite{liu2021swintransformer}.  
مشابه معماری‌های کانولوشنی، با افزایش عمق \footnote{\lr{Depth}}، ابعاد فضایی کاهش و تعداد کانال‌ها افزایش پیدا می‌کند.

در انتهای \lr{Stage} آخر، خروجی به یک لایهٔ \lr{FC} داده می‌شود تا تعداد کلاس‌ها را پیش‌بینی کند.  
پس از گذر از \lr{Softmax}، احتمال هر کلاس به‌دست می‌آید و مدل در نهایت کلاس نهایی را برمی‌گزیند.







\chapter{پیشینه پژوهش}



\subsection*{استفاده از روش‌های تانسوری در شبکه‌های عصبی چندلایه (MLP)}

در سال‌های اخیر، استفاده از روش‌های تانسوری به‌عنوان روشی نوین در بهینه‌سازی معماری‌های شبکه‌های عصبی، به‌ویژه در مدل‌هایی که تعداد پارامترهای آن‌ها بسیار زیاد است، مانند شبکه‌های عصبی چندلایه \footnote{\lr{Mlp}}، توجه بسیاری را به خود جلب کرده است. تانسورها تعمیمی از ماتریس‌ها به ابعاد بالاتر هستند و به‌طور طبیعی برای نمایش داده‌های چندبعدی همچون تصاویر، ویدیوها یا سری‌های زمانی چندکاناله مناسب‌اند. بهره‌گیری از ساختار تانسوری در معماری شبکه، این امکان را فراهم می‌سازد که بدون نیاز به فشرده‌سازی اولیه (مانند \lr{flatten} کردن ورودی)، اطلاعات ساختاری میان ابعاد مختلف حفظ شده و مدل بتواند از روابط درون‌تعاملی موجود میان این ابعاد بهره‌برداری نماید.

در معماری سنتی MLP، نگاشت از ورودی \lr{$\mathbf{x} \in \mathbb{R}^n$} به خروجی \lr{$\mathbf{y} \in \mathbb{R}^m$} به‌وسیله ضرب ماتریسی انجام می‌گیرد:

\[
\mathbf{y} = W\mathbf{x} + \mathbf{b}
\]

که در آن \lr{$W \in \mathbb{R}^{m \times n}$} ماتریس وزن و \lr{$\mathbf{b}$} بردار بایاس است. 




\begin{figure}[h]
	\centering
	\begin{minipage}[b]{0.8\textwidth}
		\centering
		\includegraphics[width=\textwidth]{transformer_images/mlp.png}
		\caption{\lr{Mlp}}
		\label{fig:Mlp}
	\end{minipage}
	\hfill
\end{figure}





در مقابل، در روش‌های تانسوری، وزن‌ها به‌صورت یک تانسور مرتبه بالاتر مدل‌سازی می‌شوند و نگاشت ورودی به خروجی با استفاده از ضرب‌های چندحالته (ضرب تانسوری در ابعاد مختلف) صورت می‌پذیرد:

\[
\mathbf{y} = \mathcal{W} \times_1 \mathbf{x}_1 \times_2 \mathbf{x}_2 \times_3 \cdots + \mathbf{b}
\]

در این رابطه، \lr{$\mathcal{W}$} یک تانسور وزن است و عملگر \lr{$\times_n$} نشان‌دهنده ضرب تانسوری در بعد $n$‌ام می‌باشد.

روش‌هایی نظیر \lr{TCL} \footnote{tensor contraction layer} و \lr{TRL} \footnote{\lr{Tensor regression layer}} به‌عنوان نمونه‌هایی از این رویکرد، با بهره‌گیری از تکنیک‌های تجزیه تانسوری همچون \lr{Tucker} یا \lr{CP decomposition}، نه تنها باعث کاهش چشمگیر در تعداد پارامترها می‌شوند، بلکه ساختار چندبعدی داده‌ها را نیز حفظ می‌نمایند. این ویژگی به‌خصوص در مسائل دارای ورودی‌های دارای ساختار فضایی یا زمانی قابل‌توجه، بسیار حائز اهمیت است.


\subsection*{مزایای استفاده از روش‌های تانسوری}

استفاده از روش‌های تانسوری در شبکه‌های عصبی چندلایه (MLP) مزایای متعددی به همراه دارد که برخی از مهم‌ترین آن‌ها عبارت‌اند از:

\begin{itemize}
	\item \textbf{کاهش چشمگیر تعداد پارامترها:} با بهره‌گیری از فشرده‌سازی تانسوری، می‌توان ابعاد تانسور وزن‌ها را به‌گونه‌ای کاهش داد که بدون افت محسوس در عملکرد مدل، مصرف حافظه و پیچیدگی محاسباتی به‌طور قابل توجهی کاهش یابد.
	
	\item \textbf{حفظ ساختار داده‌های ورودی:} برخلاف روش‌های سنتی که در آن‌ها داده‌ها پیش از ورود به لایه‌های چگال (\lr{Dense}) باید مسطح‌سازی (\lr{Flatten}) شوند، استفاده از ساختار تانسوری این امکان را فراهم می‌کند که ساختار فضایی، زمانی یا کانالی داده‌ها حفظ شده و ارتباط میان ابعاد مختلف ورودی بهتر درک و پردازش شود.
	

	
\end{itemize}


\subsection*{محدودیت‌ها و چالش‌ها}

با وجود مزایای متعدد، بهره‌گیری از روش‌های تانسوری در معماری‌های شبکه‌های عصبی با چالش‌ها و محدودیت‌هایی نیز همراه است که در ادامه به برخی از مهم‌ترین آن‌ها اشاره می‌شود:

\begin{itemize}
	\item \textbf{پیچیدگی بالاتر در پیاده‌سازی:} پیاده‌سازی لایه‌های مبتنی بر عملیات تانسوری معمولاً به ابزارها و کتابخانه‌های خاصی همچون \lr{Tensorly} یا \lr{Tensor Toolbox} نیاز دارد. این موضوع فرآیند طراحی و توسعه مدل را پیچیده‌تر از استفاده از لایه‌های استاندارد مانند \lr{Dense} یا \lr{Conv} می‌سازد.
	
	\item \textbf{بهینه‌سازی دشوارتر:} فرآیند آموزش مدل‌های تانسوری می‌تواند نسبت به مدل‌های معمولی کندتر باشد. الگوریتم‌های مبتنی بر گرادیان ممکن است در فضای پارامتری تانسورها با سطوح خطای غیرهموار یا چندوجهی مواجه شوند که روند همگرایی را دشوار می‌کند.
	
	\item \textbf{احتمال کاهش دقت در فشرده‌سازی شدید:} در صورتی‌که میزان فشرده‌سازی تانسورها بیش از حد بالا باشد، مدل ممکن است توانایی لازم برای نمایش روابط غیرخطی و الگوهای پیچیده را از دست داده و در نتیجه، دقت نهایی پیش‌بینی کاهش یابد.
\end{itemize}



\subsection{لایه فشرده‌سازی تانسوری (\lr{Tensor Contraction Layer})}

در بسیاری از مدل‌های یادگیری عمیق، به‌ویژه در شبکه‌های عصبی کانولوشنی، فعال‌سازی‌های لایه‌های میانی به‌صورت تانسورهایی با مرتبه بالا ظاهر می‌شوند. به‌طور سنتی، برای اعمال لایه‌های \lr{Fully Connected}، ابتدا این تانسورها با عملیات \lr{flattening} به بردار تبدیل شده و سپس به فضای خروجی نگاشت داده می‌شوند. این فرآیند هرچند رایج است، اما موجب از بین رفتن ساختار چندخطی (\lr{multilinear structure}) داده می‌شود و همچنین منجر به افزایش شدید تعداد پارامترهای شبکه می‌گردد.

برای مقابله با این چالش، لایه‌ای با عنوان \textbf{لایه فشرده‌سازی تانسوری} یا به اختصار \lr{TCL} معرفی شده است. این لایه بدون نیاز به \lr{flatten} کردن تانسور ورودی، ابعاد آن را در هر \lr{mode} کاهش داده و در عین حال ساختار تانسوری داده را حفظ می‌کند.


\begin{figure}[h]
	\centering
	\begin{minipage}[b]{0.7\textwidth}
		\centering
		\includegraphics[width=\textwidth]{transformer_images/tcl.png}
		\caption{\lr{tensor contraction layer}}
		\label{fig:tensor_contraction_layer}
	\end{minipage}
	\hfill
\end{figure}




\subsubsection*{فرمول‌بندی ریاضی}

فرض شود تانسور فعال‌سازی ورودی به TCL به‌صورت زیر تعریف شده باشد:

\[
\mathcal{X} \in \mathbb{R}^{S \times I_0 \times I_1 \times \cdots \times I_N}
\]

که در آن $S$ اندازه‌ی دسته‌ی آموزشی (\lr{batch size}) و $I_0, I_1, \dots, I_N$ ابعاد تانسور هستند (برای مثال عرض، ارتفاع و کانال‌های رنگی یک تصویر).

هدف لایه TCL، کاهش هر یک از ابعاد $I_k$ به $R_k$ با استفاده از ماتریس‌های فشرده‌سازی قابل آموزش به فرم زیر است:

\[
V^{(k)} \in \mathbb{R}^{R_k \times I_k}, \quad \text{برای } k = 0, 1, \dots, N
\]

عملیات فشرده‌سازی نهایی با ضرب‌های چندحالته (\lr{n-mode product}) به‌صورت زیر انجام می‌شود:

\[
\mathcal{X}' = \mathcal{X} \times_1 V^{(0)} \times_2 V^{(1)} \cdots \times_{N+1} V^{(N)}
\]

که در آن $\times_n$ نشان‌دهنده‌ی ضرب تانسور در مد $n$ام است. خروجی لایه TCL، یعنی $\mathcal{X}'$، یک تانسور فشرده‌شده با ابعاد زیر خواهد بود:

\[
\mathcal{X}' \in \mathbb{R}^{S \times R_0 \times R_1 \times \cdots \times R_N}
\]

بدین ترتیب، به‌جای تخت‌سازی و از بین رفتن ساختار چندبعدی داده، عملیات فشرده‌سازی در هر مد به‌صورت مستقل و ساختارمند انجام می‌شود.







\subsubsection*{تحلیل تعداد پارامترها}

استفاده از TCL منجر به کاهش قابل توجه در تعداد پارامترهای مدل می‌شود. تعداد پارامترهای موردنیاز برای این لایه برابر است با:

\[
\text{تعداد پارامترهای TCL} = \sum_{k=0}^{N} I_k \cdot R_k
\]

در حالی که در یک لایه \lr{Fully Connected} کلاسیک، پس از \lr{flatten} کردن ورودی، تعداد پارامترها برابر خواهد بود با:

\[
\text{تعدا پارامتر های FC} = \left( \prod_{k=0}^{N} I_k \right) \cdot O
\]

که در آن $O$ تعداد نرون‌های خروجی لایه است. همان‌گونه که پیداست، ساختار TCL باعث جایگزینی ضرب با جمع در فرمول محاسبه‌ی پارامترها می‌شود، که این امر منجر به کاهش چشمگیر حافظه و هزینه محاسباتی مدل می‌گردد.







\subsection{لایه رگرسیون تانسوری (\lr{Tensor Regression Layer})}

این لایه به‌صورت مستقیم و بدون نیاز به \lr{flatten} کردن، تانسورهای با مرتبه بالا را به بردار خروجی مدل نگاشت می‌دهد؛ به‌گونه‌ای که ساختار چندبعدی داده حفظ شده و نگاشت خروجی نیز به‌صورت \lr{low-rank} مدل‌سازی می‌شود.

\subsubsection*{فرمول‌بندی ریاضی}

فرض شود تانسور ورودی به \lr{TRL} به‌صورت زیر باشد:

\[
\mathcal{X} \in \mathbb{R}^{S \times I_0 \times I_1 \times \cdots \times I_N}
\]

که در آن $S$ اندازه‌ی دسته‌ی آموزشی \lr{(batch size) }و $I_k$ ابعاد تانسور هستند.

هدف لایه TRL نگاشت این تانسور به یک بردار خروجی $Y \in \mathbb{R}^{S \times O}$ است، که در آن $O$ تعداد نرون‌های خروجی است. نگاشت خطی میان تانسور ورودی و خروجی با استفاده از ضرب درونی تعمیم‌یافته صورت می‌گیرد:

\[
\mathbf{Y} = \langle \mathcal{X}, \mathcal{W} \rangle_N + \mathbf{b}
\]

که در آن:

\begin{itemize}
	\item $\mathcal{W} \in \mathbb{R}^{I_0 \times I_1 \times \cdots \times I_N \times O}$ تانسور وزن‌های رگرسیون است.
	\item نماد $\langle \cdot, \cdot \rangle_N$ نشان‌دهنده‌ی ضرب داخلی بر روی $N$ بعد اول از $\mathcal{W}$ و $N$ بعد آخر از $\mathcal{X}$ است.
	\item $\mathbf{b} \in \mathbb{R}^{O}$ بردار بایاس است.
\end{itemize}



\begin{figure}[h]
	\centering
	\begin{minipage}[b]{0.7\textwidth}
		\centering
		\includegraphics[width=\textwidth]{transformer_images/trl.png}
		\caption{\lr{tensor regression network}}
		\label{fig:tensor_regression_network}
	\end{minipage}
	\hfill
\end{figure}





برای کنترل تعداد پارامترها و بهره‌گیری از ساختار تانسوری، تانسور $\mathcal{W}$ با استفاده از تجزیه \lr{Tucker} مدل‌سازی می‌شود:

\[
\mathcal{W} = \mathcal{G} \times_0 U^{(0)} \times_1 U^{(1)} \cdots \times_N U^{(N)} \times_{N+1} U^{(N+1)}
\]

که در آن:

\begin{itemize}
	\item $\mathcal{G} \in \mathbb{R}^{R_0 \times R_1 \cdots \times R_N \times R_{N+1}}$ هسته‌ی کم‌مرتبه (\lr{core tensor}) است.
	\item $U^{(k)} \in \mathbb{R}^{I_k \times R_k}$ ماتریس‌های فشرده‌سازی برای ورودی هستند.
	\item $U^{(N+1)} \in \mathbb{R}^{O \times R_{N+1}}$ ماتریس فشرده‌سازی برای خروجی است.
\end{itemize}

فرمول نهایی خروجی مدل با جایگذاری $\mathcal{W}$ به‌صورت زیر خواهد بود:

\[
\mathbf{Y} = \left\langle \mathcal{X}, \mathcal{G} \times_0 U^{(0)} \cdots \times_N U^{(N)} \times_{N+1} U^{(N+1)} \right\rangle_N + \mathbf{b}
\]

یا به‌صورتی معادل و محاسباتی بهینه‌تر:

\[
\mathbf{Y} = \left\langle \mathcal{X} \times_0 (U^{(0)})^\top \cdots \times_N (U^{(N)})^\top, \mathcal{G} \times_{N+1} U^{(N+1)} \right\rangle_N + \mathbf{b}
\]

\subsubsection*{تحلیل تعداد پارامترها}

در لایه \lr{Fully Connected} سنتی، پس از \lr{flatten} کردن ورودی، تعداد پارامترها برابر است با:

\[
\text{تعداد پارامترهای FC} = \left( \prod_{k=0}^{N} I_k \right) \cdot O
\]

اما در \lr{TRL}، با در نظر گرفتن تجزیه \lr{Tucker}، تعداد پارامترها به‌صورت زیر محاسبه می‌شود:

\[
\text{تعداد پارامترهای TRL} = \left( \prod_{k=0}^{N+1} R_k \right) + \left( \sum_{k=0}^{N} I_k \cdot R_k \right) + R_{N+1} \cdot O
\]

که معمولاً به‌طور چشم‌گیری کمتر از مدل سنتی است، به‌ویژه زمانی که مقادیر $R_k$ کوچک‌تر از $I_k$ انتخاب شوند.

\subsection{چرا در مبدل های بینایی از تانسور استفاده میکنیم؟}




\subsubsection*{1. \textbf{کاهش تعداد پارامترها و حافظه مصرفی}}

یکی از چالش‌های اصلی در معماری‌های مبتنی بر \lr{Vision Transformer (ViT)}، رشد نمایی تعداد پارامترها در لایه‌های \lr{Fully Connected} به‌ویژه در بخش‌های انتهایی شبکه است.

با جایگزینی این لایه‌ها با ساختارهای تانسوری مانند \lr{Tensor Contraction Layer (TCL)} و \lr{Tensor Regression Layer (TRL)}، می‌توان از ساختار چندبعدی داده بهره برده و از طریق تجزیه‌های کم‌مرتبه، تعداد پارامترها را به‌صورت چشمگیری کاهش داد. این جایگزینی نه‌تنها سبب صرفه‌جویی در حافظه می‌شود، بلکه پیچیدگی محاسباتی مدل را نیز کاهش داده و امکان به‌کارگیری آن را در محیط‌های کم‌منبع (مانند دستگاه‌های لبه‌ای و موبایل) فراهم می‌سازد.


\subsubsection*{\textbf{افزایش تفسیرپذیری با حفظ ساختار چندخطی داده}}

در حالی‌که عملیات \lr{flattening} روی تانسورهای ورودی منجر به از بین رفتن روابط ساختاری میان ابعاد مختلف داده می‌شود، استفاده از لایه‌های تانسوری مانند \lr{Tensor Contraction Layer (TCL)} و \lr{Tensor Regression Layer (TRL)}، این امکان را فراهم می‌سازد که ساختار چندبعدی داده حفظ شود. با حفظ این ساختار چندخطی، مدل قادر خواهد بود تا وابستگی‌ها و تعاملات میان ابعاد مختلف تصویر (مانند فضا، زمان و کانال‌های رنگی) را بهتر تحلیل کند.

این ویژگی به‌طور خاص در کاربردهایی که نیازمند تفسیرپذیری بالا هستند—مانند سیستم‌های تشخیص پزشکی، بینایی ماشین صنعتی، یا کاربردهای قانونی—می‌تواند نقش مهمی ایفا کند. زیرا در چنین حوزه‌هایی، درک تصمیمات مدل توسط انسان اهمیت بالایی دارد و تحلیل ساختار داخلی مدل بر پایه روابط تانسوری، درک بهتری از رفتار مدل فراهم می‌سازد.


\subsubsection*{3. \textbf{کاهش نیاز به داده‌های آموزشی بزرگ}}

مدل‌های \lr{Vision Transformer (ViT)} به‌دلیل فقدان سوگیری مکانی ذاتی (مانند آنچه در شبکه‌های کانولوشنی وجود دارد)، نیازمند حجم عظیمی از داده برای آموزش مؤثر هستند. این مسئله در شرایطی که داده‌های برچسب‌خورده محدود هستند، به یک چالش جدی تبدیل می‌شود.

با استفاده از لایه‌های تانسوری مانند \lr{TCL} و \lr{TRL}، که نگاشت‌ها را به‌صورت چندخطی و فشرده مدل‌سازی کرده و ساختار درونی داده را حفظ می‌کنند، می‌توان از ظرفیت مدل به‌صورت بهینه‌تر بهره برد. این ساختار نه‌تنها به مدل اجازه می‌دهد که وابستگی‌های میان‌بعدی را بهتر درک کند، بلکه از بیش‌برازش در شرایط داده‌ی کم جلوگیری می‌نماید.

در نتیجه، معماری‌های مبتنی بر تانسور قادرند در دیتاست‌های کوچک نیز عملکردی قابل قبول داشته باشند و نیاز به حجم عظیم داده‌های آموزشی را کاهش دهند.



\subsection{روش تانسوری مبدل پنجره متحرک:}












\chapter{نتایج و تحلیل‌ها}

\section{ارزیابی بر روی مجموعه‌داده \lr{CIFAR-10}}

به منظور ارزیابی عملکرد مدل پیشنهادی، دو پیکربندی مجزا مورد مقایسه قرار گرفتند. در پیکربندی نخست از مدل مبنا یعنی \lr{Tiny Swin Transformer} به عنوان معماری اصلی بدون تغییرات استفاده شده است. در پیکربندی دوم، نسخه‌ی پیشنهادی یعنی \lr{Tensorized Swin Transformer} به کار گرفته شده که در آن لایه‌های \lr{Patch Embedding}، \lr{W-MSA} و \lr{Patch Merging} با بهره‌گیری از تکنیک فشرده‌سازی تانسوری بازطراحی شده‌اند.  

\subsection{خلاصه نتایج کمی}

جدول \ref{tab:cifar10_summary_tensor} نتایج کمی این دو مدل را در شاخص‌های دقت \lr{Top-1} و \lr{Top-5} برای داده‌های آموزش و آزمون نشان می‌دهد.  
در این جدول ترتیب ستون‌ها به‌گونه‌ای تنظیم شده است که مقایسه میان عملکرد آموزشی و آزمونی به‌صورت هم‌زمان قابل مشاهده باشد.

\begin{table}[ht]
	\centering
	\caption{مقایسه عملکرد مدل اصلی و مدل پیشنهادی بر روی مجموعه‌داده \lr{CIFAR-10} بر حسب دقت‌های \lr{Top-1} و \lr{Top-5}.}
	\label{tab:cifar10_summary_tensor}
	\begin{tabular}{ccccccl}
		\hline
		\multicolumn{2}{c}{داده آزمون} & \multicolumn{2}{c}{داده آموزش} & \multirow{2}{*}{\#پارامترها} & \multirow{2}{*}{مدل} \\
		\cline{1-4}
		Top-5 & Top-1 & Top-5 & Top-1 &  &  \\
		\hline
		\lr{96.45\%} & \lr{80.92\%} & \lr{99.97\%} & \lr{97.48\%} & \lr{27,528,690} & \lr{Tiny Swin} \\
		\lr{99.21\%} & \lr{81.80\%} & \lr{98.97\%} & \lr{80.30\%} & \lr{1,368,626} & \lr{Tensorized Swin} \\
		\hline
	\end{tabular}
\end{table}

\subsection{تحلیل نتایج}

مطابق جدول \ref{tab:cifar10_summary_tensor}، مدل پیشنهادی با وجود کاهش چشم‌گیر تعداد پارامترها (از حدود \lr{27.5M} به حدود \lr{1.37M}، معادل کاهش \lr{95\%})، توانسته است دقت آزمون را حفظ یا حتی اندکی بهبود بخشد. در شاخص \lr{Top-1}، دقت مدل اصلی برابر \lr{80.92\%} بوده است در حالی که مدل تانسوری به \lr{81.80\%} رسیده است. همچنین در شاخص \lr{Top-5} نیز بهبود محسوسی مشاهده می‌شود، به‌گونه‌ای که دقت از \lr{96.45\%} به \lr{99.21\%} افزایش یافته است.  

از نظر تعمیم‌پذیری نیز وضعیت مدل‌ها متفاوت است. مدل اصلی با اختلاف \lr{16.56} واحد درصد میان دقت آموزشی (\lr{97.48\%}) و آزمونی (\lr{80.92\%}) دچار بیش‌برازش بوده است. در مقابل، مدل تانسوری نه تنها دچار این مشکل نشده بلکه دقت آزمون آن اندکی بالاتر از دقت آموزش قرار گرفته است (اختلاف \lr{-1.5} واحد درصد). این نتیجه بیانگر وجود نوعی منظم‌سازی ذاتی ناشی از فشرده‌سازی تانسوری و اعمال قیود کم‌مرتبه است. همچنین، بهبود قابل توجه در شاخص \lr{Top-5} نشان می‌دهد که مدل پیشنهادی فضای ویژگی غنی‌تری ایجاد کرده است و توانایی پوشش کلاس‌های صحیح در میان پنج پیش‌بینی برتر را دارد.

\subsection{نمایش روند آموزش}

برای درک بهتر فرایند همگرایی، روند تغییرات دقت \lr{Top-1} در طول آموزش برای هر دو مدل در شکل‌های \ref{fig:cifar10_swin_original} و \ref{fig:cifar10_tensorized} نمایش داده شده است. این نمودارها نشان می‌دهند که مدل اصلی به سرعت به دقت بالایی در داده آموزشی می‌رسد ولی در داده آزمون افت می‌کند، در حالی که مدل تانسوری با روندی یکنواخت‌تر و پایدارتر به دقت نهایی دست پیدا می‌کند.

\begin{figure}[ht]
	\centering
	\includegraphics[width=0.85\textwidth]{transformer_images/results/cifar10_swin_original.png}
	\caption{روند تغییرات دقت \lr{Top-1} مدل اصلی \lr{Swin-Tiny} بر روی مجموعه‌داده \lr{CIFAR-10}.}
	\label{fig:cifar10_swin_original}
\end{figure}

\begin{figure}[ht]
	\centering
	\includegraphics[width=0.85\textwidth]{transformer_images/results/cifar10_tensorized.png}
	\caption{روند تغییرات دقت \lr{Top-1} مدل پیشنهادی \lr{Tensorized Swin} بر روی مجموعه‌داده \lr{CIFAR-10}.}
	\label{fig:cifar10_tensorized}
\end{figure}

\subsection{جمع‌بندی}

در مجموع می‌توان گفت که استفاده از ساختار تانسوری در بخش‌های کلیدی مدل، علاوه بر کاهش چشم‌گیر پیچیدگی محاسباتی و نیاز حافظه، موجب بهبود تعمیم‌پذیری و کاهش بیش‌برازش نیز شده است. این امر اهمیت به‌کارگیری روش‌های فشرده‌سازی ساختاریافته را در طراحی مدل‌های کارا برای کاربردهای کم‌منبع تأیید می‌کند.

% ======================================================
\section{نتایج بر روی دیتاست \lr{MNIST}}

برای ارزیابی عملکرد مدل پیشنهادی بر روی داده‌های ساده‌تر و با ابعاد کوچک‌تر، آزمایش‌ها بر روی دیتاست \lr{MNIST} انجام شده است. در این بخش نیز دو پیکربندی مقایسه شده‌اند: مدل اصلی \lr{Tiny Swin Transformer} و نسخه‌ی پیشنهادی یعنی \lr{Tensorized Swin Transformer}.  

\subsection{خلاصه نتایج کمی}

\begin{table}[ht]
	\centering
	\caption{مقایسه‌ی عملکرد مدل اصلی و مدل تانسوری بر روی \lr{MNIST} (فقط Top-1 و Top-5).}
	\label{tab:mnist_summary_tensor}
	\begin{tabular}{ccccccl}
		\hline
		\multicolumn{2}{c}{تست} & \multicolumn{2}{c}{آموزش} & \multirow{2}{*}{\#پارامترها} & \multirow{2}{*}{مدل} \\
		\cline{1-4}
		Top-5 & Top-1 & Top-5 & Top-1 &  &  \\
		\hline
		\lr{99.9\%} & \lr{97.0\%} & \lr{99.9\%} & \lr{95.8\%} & \lr{27,528,690} & \lr{Tiny Swin} \\
		\lr{100\%} & \lr{98.9\%} & \lr{99.9\%} & \lr{97.3\%} & \lr{1,368,626} & \lr{Tensorized Swin} \\
		\hline
	\end{tabular}
\end{table}

\subsection{نمایش روند آموزش}

شکل‌های \ref{fig:mnist_swin_original} و \ref{fig:mnist_tensorized} روند تغییرات دقت \lr{Top-1} را برای مدل اصلی و مدل تانسوری بر روی مجموعه‌داده‌ی \lr{MNIST} نشان می‌دهند. هر دو مدل به سرعت به دقت بسیار بالا رسیده‌اند، اما مدل تانسوری با وجود تعداد پارامترهای بسیار کمتر، دقت نهایی بالاتری در داده‌های آزمون کسب کرده است.

\begin{figure}[ht]
	\centering
	\includegraphics[width=0.85\textwidth]{transformer_images/results/mnist_original.png}
	\caption{روند دقت \lr{Top-1} مدل اصلی \lr{Swin-Tiny} بر روی \lr{MNIST}.}
	\label{fig:mnist_swin_original}
\end{figure}

\begin{figure}[ht]
	\centering
	\includegraphics[width=0.85\textwidth]{transformer_images/results/mnist_tensorized.png}
	\caption{روند دقت \lr{Top-1} مدل تانسوری پیشنهادی بر روی \lr{MNIST}.}
	\label{fig:mnist_tensorized}
\end{figure}

\subsection{تحلیل و بحث}

نتایج جدول \ref{tab:mnist_summary_tensor} نشان می‌دهد که مدل تانسوری تنها با \lr{1,368,626} پارامتر توانسته است عملکردی بهتر از مدل اصلی با بیش از \lr{27M} پارامتر ارائه دهد. دقت آزمون در شاخص \lr{Top-1} از \lr{97.0\%} به \lr{98.9\%} افزایش یافته و در شاخص \lr{Top-5} نیز از \lr{99.9\%} به \lr{100\%} رسیده است.  

علاوه بر این، شکاف میان دقت آموزش و آزمون در مدل اصلی حدود \lr{1.2} واحد درصد بوده است، در حالی که در مدل تانسوری این شکاف منفی شده و دقت آزمون بالاتر از دقت آموزش قرار گرفته است. این نشان‌دهنده تعمیم‌پذیری بهتر مدل تانسوری است. حتی در مجموعه‌داده‌ای ساده مانند \lr{MNIST} نیز این بهبود اهمیت دارد، زیرا تأکید می‌کند که فشرده‌سازی تانسوری نه تنها پارامترها را کاهش می‌دهد، بلکه می‌تواند به‌عنوان یک منظم‌ساز مؤثر عمل کند.

\subsection{جمع‌بندی}

به طور کلی، نتایج آزمایش‌ها بر روی دیتاست \lr{MNIST} نشان دادند که مدل تانسوری با کاهش چشمگیر پارامترها نه تنها عملکرد مدل را حفظ کرده بلکه در داده‌های آزمون نیز عملکرد بهتری داشته است. این نتایج اهمیت استفاده از روش‌های تانسوری را حتی در داده‌های ساده تأیید می‌کند.

% ======================================================
\section{نتایج بر روی دیتاست \lr{Tiny ImageNet}}

برای بررسی عملکرد مدل در یک سناریوی چالش‌برانگیزتر، آزمایش‌هایی بر روی دیتاست \lr{Tiny ImageNet} انجام شد. در این بخش علاوه بر مدل اصلی \lr{Tiny Swin Transformer}، مدل پیشنهادی تانسوری نیز با دو بهینه‌ساز متفاوت (\lr{Adam} و \lr{AdamW}) مورد ارزیابی قرار گرفته است.

\subsection{خلاصه نتایج کمی}

\begin{table}[ht]
	\centering
	\caption{مقایسه‌ی عملکرد مدل‌ها بر روی \lr{Tiny ImageNet}.}
	\label{tab:tinyimagenet_summary}
	\begin{tabular}{ccccccc}
		\hline
		\multicolumn{2}{c}{آزمون} & \multicolumn{2}{c}{آموزش} & \#پارامتر & مدل & بهینه‌ساز \\
		\cline{1-4}
		Top-5 & Top-1 & Top-5 & Top-1 &  &  &  \\
		\hline
		\lr{85.42\%} & \lr{64.14\%} & \lr{59.03\%} & \lr{42.26\%} & \lr{27,528,690} & \lr{Tiny Swin} & -- \\
		\lr{54.17\%} & \lr{30.85\%} & \lr{98.50\%} & \lr{83.89\%} & \lr{1,368,626} & \lr{Tensorized Swin} & \lr{Adam} \\
		\lr{62.07\%} & \lr{35.90\%} & \lr{80.90\%} & \lr{55.30\%} & \lr{1,368,626} & \lr{Tensorized Swin} & \lr{AdamW} \\
		\hline
	\end{tabular}
\end{table}

\subsection{نمایش روند آموزش}

شکل \ref{fig:tiny_original_top1} روند تغییر دقت \lr{Top-1} را برای مدل اصلی \lr{Tiny Swin} نشان می‌دهد. همچنین، شکل‌های \ref{fig:tiny_tensor_adam} و \ref{fig:tiny_tensor_adamw} مربوط به مدل تانسوری با بهینه‌سازهای \lr{Adam} و \lr{AdamW} هستند.

\begin{figure}[ht]
	\centering
	\includegraphics[width=0.85\textwidth]{transformer_images/results/tiny_image_net_original.png}
	\caption{روند دقت \lr{Top-1} مدل اصلی \lr{Swin-Tiny} بر روی \lr{Tiny ImageNet}.}
	\label{fig:tiny_original_top1}
\end{figure}

\begin{figure}[ht]
	\centering
	\includegraphics[width=0.85\textwidth]{transformer_images/results/tiny_image_net_tensorized_2.png}
	\caption{روند دقت \lr{Top-1} مدل تانسوری با بهینه‌ساز \lr{Adam} بر روی \lr{Tiny ImageNet}.}
	\label{fig:tiny_tensor_adam}
\end{figure}

\begin{figure}[ht]
	\centering
	\includegraphics[width=0.85\textwidth]{transformer_images/results/tiny_image_net_tensorized.png}
	\caption{روند دقت \lr{Top-1} مدل تانسوری با بهینه‌ساز \lr{AdamW} بر روی \lr{Tiny ImageNet}.}
	\label{fig:tiny_tensor_adamw}
\end{figure}

\subsection{تحلیل و بحث}

نتایج جدول \ref{tab:tinyimagenet_summary} و نمودارهای مربوطه نشان می‌دهد که رفتار مدل‌ها در این دیتاست پیچیده‌تر متفاوت از دو دیتاست قبلی است. مدل اصلی \lr{Tiny Swin} با داشتن بیش از \lr{27M} پارامتر، دقت آزمون \lr{Top-1} معقولی معادل \lr{64.14\%} به دست آورده است. با این حال اختلاف میان دقت آموزش (\lr{42.26\%}) و آزمون بسیار زیاد است که می‌تواند ناشی از محدودیت در داده‌ها یا تنظیمات بهینه‌سازی باشد.  

مدل تانسوری با بهینه‌ساز \lr{Adam} عملکردی متفاوت نشان داده است. این مدل در داده‌های آموزش به دقت بسیار بالایی (\lr{83.89\%}) رسیده اما در داده‌های آزمون تنها \lr{30.85\%} کسب کرده است، که نشانگر بیش‌برازش شدید است. در مقابل، استفاده از بهینه‌ساز \lr{AdamW} منجر به بهبود نسبی عملکرد شده و دقت آزمون به \lr{35.90\%} افزایش یافته است، هرچند همچنان فاصله‌ی زیادی تا عملکرد مدل اصلی دارد.  

این یافته‌ها بیانگر آن است که کاهش شدید پارامترها در دیتاست‌های پیچیده‌تر مانند \lr{Tiny ImageNet}، در صورت عدم استفاده از منظم‌سازی و استراتژی‌های بهینه‌سازی مناسب، می‌تواند به افت تعمیم‌پذیری منجر شود. 

\subsection{جمع‌بندی}

در مجموع، در حالی که نتایج به‌دست‌آمده روی دو دیتاست \lr{CIFAR-10} و \lr{MNIST} نشان دادند که مدل تانسوری می‌تواند همزمان با کاهش چشمگیر پارامترها عملکرد بهتری نیز ارائه دهد، در دیتاست چالش‌برانگیزتر \lr{Tiny ImageNet} این کاهش ظرفیت موجب افت دقت آزمون شده است. با این حال استفاده از بهینه‌ساز مناسب مانند \lr{AdamW} توانسته است تا حدی شکاف میان آموزش و آزمون را کاهش دهد و مسیری برای تحقیقات آینده در ترکیب روش‌های فشرده‌سازی و بهینه‌سازی پیشنهاد کند.

\include{chapter5}
\include{references}
\small
\csname@twosidefalse\endcsname
\pagestyle{fancy}
\fancyhf{} 
\fancyhead[RE]{\rightmark}
\fancyhead[LO]{\leftmark}
\fancyhead[LE,RO]{\thepage}
%\setLTRbibitems
%\resetlatinfont
%\latin
%\renewcommand{\bibname}{\rl{{کتاب‌نامه}\hfill}}
%\renewcommand{\refname}{\rl{{کتاب‌نامه}\hfill}}
\setLTRbibitems
\cfoot{}
\lhead{کتاب‌نامه}
\bibliographystyle{plain}
\bibliography{reference}

}
\appendix
\chapter{جزئیات مدل‌ها و جدول پارامترها}


{\baselineskip=.75cm
%\addcontentsline{toc}{chapter}{نام‌نامه}
\thispagestyle{empty}
\chapter*{\centering{نام‌نامه}}
\markboth{نام‌نامه}{نام‌نامه}
%
%ا          
%
\noindent
%\persiangloss{}{}
\persiangloss{آلبرت}{Albert}
\persiangloss{آئرتس}{Aerts}
\persiangloss{ادی}{Eddy}
\persiangloss{اگرستی}{Agresti}
\persiangloss{اُلکین}{Olkin}
\persiangloss{استفانسکی}{Stefanski}
\persiangloss{اسكندری}{Eskandari}
\persiangloss{اشپیگل‌هالتر}{Spiegelhalter}
\persiangloss{اُكانا-ریولا}{Oca\~na-Riola}
\persiangloss{اندرسون}{Anderson}
\persiangloss{اُهارا هاینز }{O'Hara Hines}
%\persiangloss{}{}
%
%     ب   
%\persiangloss{}{}
\persiangloss{بارتولومف}{Bartholomew}
\persiangloss{بریتنر}{Breitner}
\persiangloss{بهرامی سامانی}{Bahrami Samani}
%
%
%         پ%
%\persiangloss{}{}
\persiangloss{پرز}{P\'erez-Oc\'on}
\persiangloss{پرنتیس}{Prentice}
\persiangloss{پلاکت}{Plackett}
%
%            ت
%\persiangloss{}{}
\persiangloss{تات}{Tate}
\persiangloss{تاتز}{Tutz}
\persiangloss{تی‌سای}{Tsay}
%
%            ث
%
%            ج 
%
%\persiangloss{}{}
\persiangloss{جانسون}{Johnson}
\persiangloss{جفریز}{Jeffreys}
%
%چ             
%
%\persiangloss{}{}
\persiangloss{چان}{Chan}
\persiangloss{چاو}{Chau}
\persiangloss{چتفیلد}{Chatfield}
%
%‌ ح           
% خ             
%
% د               
 %
%\persiangloss{}{}
\persiangloss{دمپستر}{Dempster}
\persiangloss{دی}{Dey}
\persiangloss{دیگل}{Diggle}
\persiangloss{دیل}{Dale}
 %
% ذ             
%
%  ر 
 %
\persiangloss{رابنستین}{Rubinstein}   
\persiangloss{رابین}{Rubin}
\persiangloss{رضایی}{Rezaee}
\persiangloss{رضایی قهرودی}{Rezaei Ghahroodi}
\persiangloss{رفتری}{Raftery}
\persiangloss{ریان}{Ryan}
\persiangloss{ریدات}{Ridout}
%\persiangloss{}{}
% 
%  ز              %
%\persiangloss{}{}
\persiangloss{زگر }{Zeger}
%
% ‌ ژ             
%
%  س            ‌%
%\persiangloss{}{}
\persiangloss{سامر}{Sommer}
\persiangloss{سانگ}{Sung}
%
%   ش      %
%\persiangloss{}{}
\persiangloss{شفر}{Schafer}
\persiangloss{شلاتچر}{Schluchter}
\persiangloss{شیه}{Shih}
%        ص       
%      ض 
%   ط           
%     ظ           
%\persiangloss{}{}
%     ع         
%     غ           
%     ف          %
%\persiangloss{}{}
\persiangloss{فایندلی}{Findlay}
\persiangloss{فرانکوم}{Francom}
\persiangloss{فلر}{Feller}
\persiangloss{فهرمير}{Fahrmeir}
\persiangloss{فیتز موریس}{Fitzmaurice}
%     ق  
% ک           %
%\persiangloss{}{}
\persiangloss{کاتالانو}{Catalano}
\persiangloss{کارول}{Carroll}
\persiangloss{کاس}{Kass}
\persiangloss{کافمن}{Kaufmann}
\persiangloss{کاکس}{Cox}
\persiangloss{کاکیش}{Qaqish}
\persiangloss{كالبفلیش}{Kalbfleisch}
\persiangloss{کروسکال}{Kruskal}
\persiangloss{كنوارد}{Kenward}
\persiangloss{کورن}{Korn}
\persiangloss{کولینز}{Collins}
% گ          %
%\persiangloss{}{}
\persiangloss{گابریل}{Gabriel}
\persiangloss{گلفند}{Gelfand}
\persiangloss{گنجعلی}{Ganjali}
\persiangloss{گود}{Good}
\persiangloss{گودمن}{Goodman}
\persiangloss{گیسر}{Geisser}
%
%
 %
%
%   ل         %
%\persiangloss{}{}
\persiangloss{لان}{Lunn}
\persiangloss{لایرد}{Laird}
\persiangloss{لاولس}{Lawless}
\persiangloss{لسافره}{Lesaffre}
\persiangloss{لیانگ}{Liang}
\persiangloss{لیپ‌شیتز}{Lipsitz}
\persiangloss{لیتل}{Little}
\persiangloss{لیندلی}{Lindley}
%
%     م        %
%\persiangloss{}{}
\persiangloss{مادیگان}{Madigan}
\persiangloss{مشكانی}{Meshkani}
\persiangloss{موری}{Murray}
\persiangloss{مولنبرگز}{Molenberghs}  
\persiangloss{موئنز}{Muenz}
\persiangloss{مه‌کولا}{McCullagh}
\persiangloss{مه‌کوله}{McCulloch}
\persiangloss{مید}{Mead}
%       ن     %
%\persiangloss{}{}
\persiangloss{ناگل‌كرك}{Nagelkerke}
\persiangloss{نلدر}{Nelder}
\persiangloss{نوریان}{Noorian}
\persiangloss{نیوتن}{Newton}
\persiangloss{نیوهاز}{Neuhaus}
%
%
%          و %
%\persiangloss{}{}
\persiangloss{ودربرن}{Wedderburn}
\persiangloss{وربک}{ٰVerbeke}
\persiangloss{ورموت}{Wermuth}
\persiangloss{وُنگ}{Wong}
\persiangloss{وو}{Wu}
\persiangloss{ویت‌مور}{Whittemore}
\persiangloss{وِیر}{Ware}
%               ه        %
%\persiangloss{}{}
%\persiangloss{هارویل}{Harville}
\persiangloss{هکمن}{Heckman}
%\persiangloss{هندرسون}{Henderson}
\persiangloss{هیتچکاک}{Hitchcock}
\persiangloss{هاینز}{Hines}
%
%            ی 
%\persiangloss{}{}
\persiangloss{یانگ}{Yang}
%  
%
%\addcontentsline{toc}{chapter}{واژه‌نامه فارسی به انگلیسی}
%\thispagestyle{empty}
\chapter*{\centering{واژه‌نامه فارسی به انگلیسی}}
\markboth{واژه‌نامه فارسی به انگلیسی}{واژه‌نامه فارسی به انگلیسی}

\noindent
%\englishgloss{}{}
\englishgloss{Cross-Over Trials}{آزمایه‌های متقاطع}
\englishgloss{Peabody Individual Achievement Test (PIAT)}{آزمون انفرادی  پيشرفت پی‌بادی}
\englishgloss{Equilibrium Probability}{احتمال  تعادل}
\englishgloss{Nelder and Mead Simplex Algorithm}{الگوریتم سادکی نلدر و مید}
\englishgloss{Insomnia}{بی‌خوابی}
\englishgloss{Parsimonious Parametrization}{پارامتریدن ممسك}
\englishgloss{Link Function}{تابع ربط}
\englishgloss{Exchangeable}{تبادل‌پذیر}
\englishgloss{Mastitis Data}{داده‌های آماس غده‌ها}
\englishgloss{Xerophthalmia}{رمد چشم}
\englishgloss{Absorbing Markov chain}{زنجیر مارکوف جاذب}
\englishgloss{Non-homogeneous Markov Chain}{زنجیر مارکوف ناهمگن}
\englishgloss{Identifiability}{شناسایی‌پذیری}
\englishgloss{Innovation}{عامل ابداع}
\englishgloss{Conditional Predictive Ordinate (CPO)}{عرض پیشگوی شرطی}
\englishgloss{Irreversible Process}{فرایند برگشت‌ناپذیر}
\englishgloss{Serially Correlated Process}{فرایند همبسته‌ی پیاپی}
\englishgloss{Quadrature Formula}{فرمول تربیعی}
\englishgloss{Fluvoxamine}{فلووکسامین}
\englishgloss{Discriminant Analysis Applications}{كاربردهای تحلیل  تشخیصی}
\englishgloss{Latent Variable}{متغیر پنهان}
\englishgloss{Bayesian Generalized Additive Mixed Model}{مدل آمیخته جمعی تعمیم‌یافته بیزی}
\englishgloss{Random-Coefficient Pattern-Mixture Models}{مدل‌های الگو  آمیخته‌ی ضرایب تصادفی}
\englishgloss{Random-Coefficient Selection Models}{مدل‌های گزینش ضرایب تصادفی}
\englishgloss{Likelihood Based Mixture Parameter Model}{مدل  پارامتر آمیخته  مبتنی بر درستنمایی}
\englishgloss{Subject-Specific Model}{مدل خاص آزمودنی}
\englishgloss{General Location Model}{مدل مکانی عام}
\englishgloss{Population-Averaged Model}{مدل  میانگین-جامعه}
\englishgloss{Generalized Heckman Model}{مدل هكمن تعمیم‌یافته}
\englishgloss{Indonesian Children's Health Study}{مطالعه سلامت کودکان اندونزیایی}
\englishgloss{Score Equations}{معادله‌های امتیاز}
\englishgloss{Generalized Estimating Equations}{معادله‌های  برآوردساز تعمیم‌یافته}
\englishgloss{Deviance Information Criterion (DIC)}{ملاک اطلاع انحرافه}
\englishgloss{Cut-Point}{نقطه‌ی برش}
\englishgloss{Change Point}{نقطه‌ی تغییر}
\englishgloss{Isotonic}{همنوا}
\englishgloss{m-Dependence}{$m$-وابستگی}
%
%\addcontentsline{toc}{chapter}{واژه‌نامه  انگلیسی به  فارسی}
\thispagestyle{empty}
\chapter*{\centering{واژه‌نامه  انگلیسی به  فارسی}}
\markboth{واژه‌نامه  انگلیسی به  فارسی}{واژه‌نامه  انگلیسی به  فارسی}

\noindent
\persiangloss{شکنندگی}{Frailty}
\persiangloss{مدل نرخ خطرهای متناسب کاکس}{Proportional Hazards Rate Model}
\persiangloss{مطالعات خانوادگی}{Familial Studies}
\persiangloss{آزمایه‌های بالینی چندمرکزی}{Multicenter clinicla trials}
\persiangloss{استقلال بین خوشه‌ای}{Intercluster Independence}
\persiangloss{شناسایی‌پذیر}{Identifiable}
\persiangloss{داده‌های بقای فضایی}{Spatial Survival Data}
\persiangloss{استقلال بین خوشه‌ای}{Intercluster Independence}
\persiangloss{استقلال بین خوشه‌ای}{Intercluster Independence}
\persiangloss{استقلال بین خوشه‌ای}{Intercluster Independence}
\persiangloss{استقلال بین خوشه‌ای}{Intercluster Independence}
\persiangloss{استقلال بین خوشه‌ای}{Intercluster Independence}
\printindex
% در این فایل، عنوان پایان‌نامه، مشخصات خود و چکیده پایان‌نامه را به انگلیسی، وارد کنید.
% توجه داشته باشید که جدول حاوی مشخصات پایان‌نامه/رساله، به طور خودکار، رسم می‌شود.
%%%%%%%%%%%%%%%%%%%%%%%%%%%%%%%%%%%%
\baselineskip=.6cm
\begin{latin}
\latintitle{Analysis of brain data using graph neural networks}
\latinuniversity{Tarbiat Modares University}
\latinfaculty{Faculty of Mathematical Sciences}
\latindegree{Master in Computer Science}
%group:
\latinsubject{Department of Mathamatiics}
\latinfield{}
\latintitle{Vision Trnasformer}
\firstlatinsupervisor{Dr. Mansoor Rezghi }
%\secondlatinsupervisor{Second Supervisor}
%\firstlatinadvisor{First Advisor}
%\secondlatinadvisor{Second Advisor}
\latinname{Seyed Mohammad}
\latinsurname{Badzohreh}
\latinthesisdate{2024}
\latinkeywords{ }

\newpage
\thispagestyle{empty}
\baselineskip=.750cm
\en-abstract{%\noindent
Some data that present the time ...
}


%\markboth{\lr{Abstract}}{\lr{Abstract}}
\noindent
\begin{large}
\begin{center}
Abstract
\end{center}
\end{large}
%

In recent years, vision transformers have emerged as one of the core architectures in deep learning models for image processing. However, conventional designs of these transformers often rely on fully connected layers that require flattening high-order input tensors into vectors. This process not only leads to a significant increase in the number of parameters but also results in the loss of spatial structures and inter-dimensional relationships within the data. In this study, we propose a tensor-based framework for the design of a window-based vision transformer that leverages Tensor Compression Layers (TCL) and Tensor Regression Layers (TRL) to effectively model multilinear mappings across data dimensions. By preserving the multidimensional structure of images, the proposed method achieves a substantial reduction in the number of parameters while maintaining structural information and improving the interpretability of the model. Experimental results on standard benchmark datasets demonstrate that incorporating tensorized structures not only reduces computational complexity but also enhances classification accuracy.
\\
\newline
{{\bf Key Words:}} Vision Transformer, Tensor Decomposition, Tensor Compression Layer (TCL), Tensor Regression Layer (TRL), Multilinear Mapping, Image Classification, Deep Learning.%}
\newpage
\latinvtitle
%
%\latintitle{Analysis of brain data using graph neural networks}
%\latinuniversity{Tarbiat Modares University}
%\latinfaculty{Faculty of Mathematical Sciences}
%\latindegree{Master in Computer Science}
%%group:
%\latinsubject{Department of Mathamatiics}
%\latinfield{}
%\latintitle{Analysis of brain data using graph neural networks}
%\firstlatinsupervisor{Mansour Rezghi Ahagh}
%%\secondlatinsupervisor{Second Supervisor}
%%\firstlatinadvisor{First Advisor}
%%\secondlatinadvisor{Second Advisor}
%\latinname{Mitra}
%\latinsurname{Nemati Andavari}
%\latinthesisdate{2024}
%\latinkeywords{ }
\end{latin}
}
\label{LastPage}
\end{document}