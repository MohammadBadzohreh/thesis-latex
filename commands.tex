% در این فایل، دستورها و تنظیمات مورد نیاز، آورده شده است.
%--------------------------------------------------------------Tarbiat Modares-----------------------------------------------------
\usepackage{algpseudocode}

\usepackage{multirow}
\usepackage{rotating}
\usepackage{lscape}
\usepackage{float}

% در ورژن جدید زی‌پرشین برای تایپ متن‌های ریاضی، این سه بسته، حتماً باید فراخوانی شود
\usepackage{amsthm,amssymb,amsmath}
% بسته‌ای برای تنطیم حاشیه‌های بالا، پایین، چپ و راست صفحه
\usepackage[top=40mm, bottom=30mm, left=25mm, right=35mm]{geometry}
\setlength{\textwidth}{15cm}
\setlength{\oddsidemargin}{2mm}
\renewcommand{\baselinestretch}{1.5}     % ꉑ¬‰Ü‰‚ ¡‰Î‰Ï‰
% بسته‌‌ای برای ظاهر شدن شکل‌ها و تصاویر متن
\usepackage{graphicx}
% بسته‌ای برای رسم کادر
%\usepackage{framed} 
% بسته‌‌ای برای چاپ شدن خودکار تعداد صفحات در صفحه «معرفی پایان‌نامه»
\usepackage{lastpage}
% بسته‌ و دستوراتی برای ایجاد لینک‌های رنگی با امکان جهش
\usepackage[linktocpage=true,colorlinks,pagebackref=true,linkcolor=blue,citecolor=magenta]{hyperref}
%\usepackage[pagebackref=false,colorlinks,linkcolor=blue,citecolor=magenta]{hyperref}
% چنانچه قصد پرینت گرفتن نوشته خود را دارید، خط بالا را غیرفعال و  از دستور زیر استفاده کنید چون در صورت استفاده از دستور زیر‌‌، 
% لینک‌ها به رنگ سیاه ظاهر خواهند شد که برای پرینت گرفتن، مناسب‌تر است
%\usepackage[pagebackref=false]{hyperref}
% بسته‌ لازم برای تنظیم سربرگ‌ها
\usepackage{fancyhdr}
% بسته‌ای برای ظاهر شدن «مراجع» و «نمایه» در فهرست مطالب
\usepackage[nottoc]{tocbibind}
% دستورات مربوط به ایجاد نمایه
\usepackage{makeidx}
\makeindex
%%%%%%%%%%%%%%%%%%%%%%%%%%
% فراخوانی بسته زی‌پرشین و تعریف قلم فارسی و انگلیسی
\usepackage{xepersian}
\settextfont[Scale=1.1]{XB Niloofar.ttf}

%%%%%%%%%%%%%%%%%%%%%%%%%%
% چنانچه می‌خواهید اعداد در فرمول‌ها، انگلیسی باشد، خط زیر را غیرفعال کنید
%\setdigitfont[Scale=0.9]{Persian Modern}
%\setdigitfont.ttf[Scale=1.1]{XB Zar}
%\setdigitfont[Scale.ttf=1.1]{B Nazanin Outline}
%%%%%%%%%%%%%%%%%%%%%%%.ttf%%%
% تعریف قلم‌های فارسی و انگلیسی اضافی بر.ttfای استفاده در بعضی از قسمت‌های متن
\defpersianfont\nastaliq[Scale=2.02]{IranNastaliq.ttf}
\defpersianfont\bnazaninout[Scale=2]{B Nazanin Outline.ttf} 
%\defpersianfont\chapternumber[Scale=2.5]{XB Kayhan Pook.ttf}
%\renewcommand{\chapternumber}[1][Scale=2.5]{\fontspec{XB Kayhan Pook.ttf}}
\defpersianfont\dav[Scale=2]{XB Kayhan Pook.ttf}
\defpersianfont\titr[Scale=1.02]{XB Titre.ttf}
\defpersianfont\chapternumber[Scale=3]{XB Niloofar.ttf}
%\renewcommand{\chapternumber}[1][Scale=3]{\fontspec{XB Niloofar.ttf}}

%%%%%%%%%%%%%%%%%%%%%%%%%%
% دستوری برای حذف کلمه «چکیده»
%\renewcommand{\abstractname}{}
% دستوری برای حذف کلمه «abstract»
\renewcommand{\latinabstract}{}
% دستوری برای تغییر نام کلمه «اثبات» به «برهان»
\renewcommand\proofname{\textbf{برهان}}
% دستوری برای تغییر نام کلمه «کتاب‌نامه» به «مراجع»
\renewcommand{\bibname}{کتاب‌نامه}
\def\contentsname{فهرست مندرجات}
\def\listfigurename{لیست اشکال}
% دستوری برای تعریف واژه‌نامه انگلیسی به فارسی
\newcommand\persiangloss[2]{#1\dotfill\lr{#2}\\}
% دستوری برای تعریف واژه‌نامه فارسی به انگلیسی 
\newcommand\englishgloss[2]{#2\dotfill\lr{#1}\\}
% تعریف دستور جدید «\پ» برای خلاصه‌نویسی جهت نوشتن عبارت «پروژه/پایان‌نامه/رساله»
\newcommand{\پ}{پروژه/پایان‌نامه/رساله }
%%%%%%%%%%%%%%%%%%%%%%%%%%
% تعریف و نحوه ظاهر شدن عنوان قضیه‌ها، تعریف‌ها، مثال‌ها و ...
\theoremstyle{definition}
\newtheorem{definition}{تعریف}[section]
\theoremstyle{theorem}
\newtheorem{theorem}[definition]{قضیه}
\newtheorem{lemma}[definition]{لم}
\newtheorem{proposition}[definition]{گزاره}
\newtheorem{corollary}[definition]{نتیجه}
\newtheorem{remark}[definition]{ملاحظه}
\theoremstyle{definition}
\newtheorem{example}[definition]{مثال}
\newcommand{\x}{\mbox{\boldmath$x$\unboldmath}}
\newcommand{\dd}{\mbox{\boldmath$d$\unboldmath}}
\newcommand{\Uu}{\mbox{\boldmath$U$\unboldmath}}
 \newcommand{\balpha}{\mbox{\boldmath$\alpha$\unboldmath}}
 \newcommand{\bbeta}{\mbox{\boldmath$\beta$\unboldmath}}
  \newcommand{\btheta}{\mbox{\boldmath$\theta$\unboldmath}}
 \newcommand{\bphi}{\mbox{\boldmath$\phi$\unboldmath}}
  \newcommand{\bpsi}{\mbox{\boldmath$\psi$\unboldmath}}
\newcommand{\bmu}{\mbox{\boldmath$\mu$\unboldmath}}
 \newcommand{\X}{\mbox{\boldmath$X$\unboldmath}}
\newcommand{\Y}{\mbox{\boldmath$Y$\unboldmath}}
\newcommand{\z}{\mbox{\boldmath$z$\unboldmath}}
 \newcommand{\W}{\mbox{\boldmath$W$\unboldmath}}
 \newcommand{\q}{\mbox{\boldmath$q$\unboldmath}}
 \newcommand{\bb}{\mbox{\boldmath$b$\unboldmath}}
 \newcommand{\I}{\mbox{\boldmath$I$\unboldmath}}
\newcommand{\w}{\mbox{\boldmath$w$\unboldmath}}
\newcommand{\tb}{\mbox{\boldmath$\theta$\unboldmath}}
\newcommand{\bSigma}{\mbox{\boldmath$\Sigma$\unboldmath}}
\newcommand{\bsigma}{\mbox{\boldmath$\sigma$\unboldmath}}
\newcommand{\logit}{{\rm{logit}}}
\newcommand{\rth}{{\rm{th}}}
\newcommand{\E}{{\rm{E}}}
\newcommand{\Var}{{\rm{Var}}}
\newcommand{\Pq}{{\rm{P}}}
\newcommand{\ICHS}{مطالعه سلامت کودکان اندونزیایی}
\newcommand{\OR}{{\rm{OR}}}
\newcommand{\Corr}{{\rm{Corr}}}
\newcommand{\Cov}{{\rm{Cov}}}
%
%%%%%%%%%%%%%%%%%%%%%%%%%%%%
% دستورهایی برای سفارشی کردن سربرگ صفحات
\csname@twosidefalse\endcsname
\pagestyle{fancy}
\fancyhf{} 
\fancyhead[RE]{\rightmark}
\fancyhead[LO]{\leftmark}
\fancyhead[LE,RO]{\thepage}
\renewcommand{\chaptermark}[1]{%
\markboth{\thechapter.\ #1}{}}
% دستورهایی برای سفارشی کردن سربرگ صفحات
%\csname@twosidetrue\endcsname
%\pagestyle{fancy}
%\fancyhf{} 
%\fancyhead[OL,EL]{\thepage}
%\fancyhead[OR]{\small\rightmark}
%\fancyhead[ER]{\small\leftmark}
%\renewcommand{\chaptermark}[1]{%
%\markboth{\thechapter.\ #1}{}}
%%%%%%%%%%%%%%%%%%%%%%%%%%%%%
% دستورهایی برای سفارشی کردن صفحات اول فصل‌ها
\makeatletter
\newcommand\mycustomraggedright{%
 \if@RTL\raggedleft%
 \else\raggedright%
 \fi}
\def\@part[#1]#2{%
\ifnum \c@secnumdepth >-2\relax
\refstepcounter{part}%
\addcontentsline{toc}{part}{\thepart\hspace{1em}#1}%
\else
\addcontentsline{toc}{part}{#1}%
\fi
\markboth{}{}%
{\centering
\interlinepenalty \@M
\ifnum \c@secnumdepth >-2\relax
 \Huge\bfseries \partname\nobreakspace\thepart
\par
\vskip 20\p@
\fi
\Huge\bfseries #2\par}%
\@endpart}

\def\@makechapterhead#1{%
\vspace*{50pt} 
%\vspace*{-30\p@}%
{\parindent \z@ \mycustomraggedright 
%{\parindent 0pt \mycustomraggedright 
{\centering{
\ifnum \c@secnumdepth >\m@ne
\if@mainmatter

%\huge\bfseries \@chapapp\space {\chapternumber\thechapter}
%\huge\bnazaninout \@chapapp\space {\chapternumber\thechapter}
{\large\bnazaninout \@chapapp{} \thechapter \par}
\par\nobreak
\vskip 20\p@
\fi
% \vskip 20pt 
\fi
\interlinepenalty\@M 
\Huge \bfseries #1\par\nobreak  \vskip 120\p@  }}}}
% \Huge \bf #1       \par \nobreak \vskip 40pt }}}}
\makeatother