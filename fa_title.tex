% در این فایل، عنوان پایان‌نامه، مشخصات خود، متن تقدیمی‌، ستایش، سپاس‌گزاری و چکیده پایان‌نامه را به فارسی، وارد کنید.
% توجه داشته باشید که جدول حاوی مشخصات پایان‌نامه/رساله و همچنین، مشخصات داخل آن، به طور خودکار، درج می‌شود.
%%%%%%%%%%%%%%%%%%%%%%%%%%%%%%%%%%%%
% دانشگاه خود را وارد کنید
\university{تربیت مدرس}
% دانشکده، آموزشکده و یا پژوهشکده  خود را وارد کنید
\faculty{علوم ریاضی}
% گروه آموزشی خود را وارد کارشناسی ارشد آمار (رشته)}
\degree {کارشناسی ارشد علوم کامپیوتر}

% گروه آموزشی خود را وارد کنید
\subject{علوم کامپیوتر }
% گرایش خود را وارد کنید
\field{داده کاوی}
% عنوان پایان‌نامه را وارد کنید
\title{روش های عمیق مبتنی برمبدل های بینایی در تحلیل داده های تصویری}
% نام استاد(ان) راهنما را وارد کنید
\firstsupervisor{آقای دکتر منصور رزقی}
%\secondsupervisor{استاد راهنمای دوم}
% نام استاد(دان) مشاور را وارد کنید. چنانچه استاد مشاور ندارید، دستور پایین را غیرفعال کنید.
%\firstadvisor{نام استاد(دان) مشاور}
%نام استاد مشاور را وارد کنید. چنانچه دو استاد مشاور دارید  دستور پایین را نیز فعال کنید.
%\secondadvisor{نام استاد(دان) مشاور}
% نام پژوهشگر را وارد کنید
\name{سید محمد}
% نام خانوادگی پژوهشگر را وارد کنید
\surname{بادزهره}
% تاریخ پایان‌نامه را وارد کنید
\thesisdate{تابستان ۱۴۰۴}
% کلمات کلیدی پایان‌نامه را وارد کنید
%\keywords{داده‌های بقا فضایی، مدل‌های شکنندگی فضایی}
% چکیده پایان‌نامه را وارد کنید
\newpage
%\fa-abstract{\noindent
%در 

%}
%\newpage
\thispagestyle{empty}
\vtitle
%\newpage
%\thispagestyle{empty}
%\clearpage
%~~~
%\newpage
%\thispagestyle{empty}
%\input{rights}
%\newpage
%\thispagestyle{empty}
%\clearpage
%~~~
%\newpage
%\thispagestyle{empty}
%\centerline{{\includegraphics[width=20 cm]{replyrecord}}}
%\newpage
%\thispagestyle{empty}
%\clearpage
%~~~
\newpage
 % پایان‌نامه خود را تقدیم کنید!
\begin{acknowledgementpage}

\vspace{4cm}

{\nastaliq
{\Large
 تقدیم به
\vspace{1.5cm}

\newdimen\xa
\xa=\textwidth
% \advance \xa by -14cm
% \hspace{\xa}
پدر بزرگوار و مادر مهربانم و برادر عزیزم 
\newline
آن ها که از خواسته هایشان گذشتند، سختی ها را به جان خریدند و خود را سپر بلای مشکلات و ناملایمات کردند تا من به جایگاهی که اکنون در آن ایستاده ام برسم .


% \vspace{1.5cm}

\newdimen\xa
\xa=\textwidth
\advance \xa by -12cm
\hspace{\xa}
%...
\vspace{1.5cm}

\newdimen\xa
\xa=\textwidth
\advance \xa by -10cm
\hspace{\xa}

\vspace{1.5cm}

\newdimen\xa
\xa=\textwidth
\advance \xa by -8cm
\hspace{\xa}
}}
\end{acknowledgementpage}
%\newpage
%\thispagestyle{empty}
%\clearpage
%~~~
%%%%%%%%%%%%%%%%%%%%%%%%%%%%%%%%%%%%
%\newpage
%\thispagestyle{empty}
% ستایش
%\baselineskip=.750cm
%\ \\ \\
% 
%{\nastaliq
%رسيدن، به دانش است و به كردار نیک...%
%}\\
%\vspace{.5cm}\\
%{\scriptsize\nastaliq
%{
% و 
% }}
% 
%\vspace{.5cm}
%{\nastaliq
%\newdimen\xb
%\xb=\textwidth
%\advance \xb by -8.5cm
%\hspace{\xb}
%پس منطق ناگزير آمد بر جوينده‌ی رستگاری.
%\RTLfootnote{مقدمه‌ی رساله‌ی منطق دانشنامه‌ی علائی، شیخ‌الرئیس ابن‌سینا}
%}
%\newpage
%\thispagestyle{empty}
%\clearpage
%~~~
%%%%%%%%%%%%%%%%%%%%%%%%%%%%%%%%%%%%
\newpage
\thispagestyle{empty}
%قدردانی
{\nastaliq
قدردانی \newline
 \newline
 
از استاد گرانقدر، جناب آقای دکتر رزقی که با راهنمایی‌های دلسوزانه و ارزشمند خود، همواره در مسیر تحقیق این پایان‌نامه یار و راهنمای من بودند، نهایت سپاس و قدردانی را دارم.
\newline

از خانواده عزیزم که با محبت بی‌پایان، صبوری و حمایت‌های بی‌دریغ‌شان، همواره پشتیبان من در طی این مسیر سخت و پرچالش بودند، صمیمانه سپاسگزارم.

}
{\scriptsize\nastaliq
%سپاس و ستایش خدایی را که هستی نام ازو یافت، فلک جنبش، زمین آرام ازو یافت.
}


% با استفاده از دستور زیر، امضای شما، به طور خودکار، درج می‌شود
\signature 
%
\newpage
\thispagestyle{empty}
\baselineskip=1.0cm
\ \\ \\
\begin{large}
\begin{center}
چکیده
\end{center}
\end{large}
%}}
%\\[4cm]


مبدل‌های بینایی در سال‌های اخیر به یکی از ساختارهای اصلی در مدل‌های یادگیری عمیق برای پردازش تصاویر تبدیل شده‌اند. با این حال، ساختارهای کلاسیک این مبدل‌ها معمولاً شامل لایه‌های کاملاً متصل هستند که نیازمند تبدیل تانسورهای چندبعدی ورودی به بردارهای مسطح می‌باشند. این فرآیند منجر به افزایش چشم‌گیر پارامترها و از بین رفتن ساختار فضایی و روابط میان‌بعدی داده می‌شود. در این پژوهش، چارچوبی تانسوری برای طراحی مدل مبدل تصویری پنجره متحرک پیشنهاد شده است که با بهره‌گیری از لایه‌های فشرده‌سازی تانسوری (\lr{TCL}) و رگرسیون تانسوری (\lr{TRL})، نگاشت‌های چندخطی میان ابعاد داده را مدل‌سازی می‌کند. این روش با حفظ ساختار چندبعدی تصویر، موجب کاهش قابل‌توجه تعداد پارامترها، حفظ اطلاعات ساختاری، و افزایش تفسیرپذیری مدل می‌شود. 
نتایج حاصل از پیاده‌سازی و ارزیابی مدل در مجموعه‌داده‌های استاندارد نشان می‌دهد که استفاده از ساختارهای تانسوری نه‌تنها پیچیدگی محاسباتی را کاهش می‌دهد، بلکه دقت طبقه‌بندی را نیز بهبود می‌بخشد.





%\newpage
%\thispagestyle{empty}
%\clearpage
%~~~
%\newpage
%{\small
%\abstractview
%}
%\newpage
%\thispagestyle{empty}
%\clearpage
%~~~
\newpage