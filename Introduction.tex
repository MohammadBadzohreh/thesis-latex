\chapter*{پیش‌گفتار}
\phantomsection
\addcontentsline{toc}{chapter}{پیش‌گفتار}



\section*{پیش‌گفتار}

مبدل‌های بینایی\footnote{\lr{vision transformers}}در سال‌های اخیر توجه زیادی در حوزه بینایی ماشین و یادگیری عمیق به خود جلب کرده‌اند. این مدل‌ها با وجود عملکرد قوی، دارای تعداد زیادی پارامتر هستند و معمولاً ساختار فضایی داده را از بین می‌برند. استفاده از روش‌های تانسوری راه‌حلی برای این مسئله ارائه می‌دهد.

در این پژوهش، چارچوبی مبتنی بر ساختارهای تانسوری برای بهینه‌سازی مبدل‌های بینایی طراحی شده است. هدف از این کار، کاهش تعداد پارامترها، حفظ ساختار داده و بهبود تفسیرپذیری مدل است. از لایه‌های فشرده‌سازی تانسوری (\lr{TCL}) و رگرسیون تانسوری (\lr{TRL}) برای مدل‌سازی نگاشت‌های چندبعدی استفاده شده است.

این پایان‌نامه شامل چهار فصل است:

- \textbf{فصل اول}: بیان مقدمه، مسئله تحقیق، اهداف پژوهش و اهمیت موضوع.
- \textbf{فصل دوم}: مرور پیشینه پژوهش، معرفی ساختار مبدل‌های بینایی و مبدل‌های پنجره متحرک.
- \textbf{فصل سوم}: تشریح روش پیشنهادی، طراحی مدل تانسوری و پیاده‌سازی معماری مبدل پنجره متحرک به صورت تانسوری.
- \textbf{فصل چهارم}: ارزیابی مدل، تحلیل نتایج حاصل از آزمایش‌ها و مقایسه عملکرد مدل پیشنهادی با مدل‌های پایه.

در پایان، جمع‌بندی، نتیجه‌گیری و پیشنهادهایی برای پژوهش‌های آینده ارائه شده است.



