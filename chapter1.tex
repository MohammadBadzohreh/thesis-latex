\chapter{مفاهیم اولیه}
در این فصل به معرفی مقدمات و مفاهیم مورد نیاز در این پایان‌نامه می‌پردازیم. 
\section{مقدمه}
در این بخش به تاریخچه هوش مصنوعی، دستاورد های اولیه، چالش ها، دلایل رکود هوش مصنوعی و پایان عصر تاریک هوش مصنوعی صحبت میکنیم 

% update the abstract


\subsection{آغاز هوش مصنوعی و هدف اصلی}

هوش مصنوعی به عنوان شاخه ای از علوم کامپیوتر، در دهه 1950 با هدف ساخت  سیستم ها و ماشین هایی که توانایی تقلد از هوش انسانی را دارند، آغاز شد.
نخستین بار مکارتی در سال 1956 این اصطلاخ را به کار گرفت. و هوش مصنوعی به عنوان علمی که در آن به مطالعه الگوریتم هایی برای تقلید رفتار انسانی می پردازد، شناخته شد.
اهداف اولیه هوش مصنوعی شامل توانایی درک زبان، یادگیری، حل مسئله و تولید موجودات هوشمند بود.
در این دوران پروژه های تحقیقاتی زیادی به امید دستیابی به هوش مصنوعی عمومی 
 AGI (Artificial General Intelligence)
شروع به کار کردند

\subsection{دورهٔ طلایی و پیشرفت‌های اولیه}
در دهه 50 و60 میلادی، هوش مصنوعی به عنوان یکی از پرچمداران پژوهش های نوین شناخته می شد.
الگوریتم های اولیه به کمک روش های منطقی و ریاضیاتی برای حل مسئله و بازی های ساده توسعه یافتند مانند انواع الگوریتم جستوجوی درختی که در این دوره به وجود آمدند و زمینه ساز اولین دستاوردهای هوش مصنوعی در بازی های تخته ای همچون شطرنج شدند.
در این دوران پیشرفت های بیشتری در پردازش زبان طبیعی (NLP)
و سیستم های خبره (Expert Systems) نیز صورت گرفت که این امید را در دانشمندان و محققان تقویت کرد که دستیابی به هوش مصنوعی عمومی به زودی ممکن خواهد بود.

\subsection{انتظارات بیش از حد و ظهور عصر تاریک}
با وجود پیشرفت های های هوش مصنوعی، محدودیت های تکنولوژی مثل gpu ها  و محاسباتی در آن زمان و همچنین کمبود داده های کافی برای آموزش مدل های پیچیده تر، باعث شد که بسیاری از پروژه های تحقیقاتی نتوانند به نتایج پیش بینی شده دست یابند. و در نتیجه، هوش مصنوعی در دهه 70 به مرحله ای از رکود وارد شد که به آن عصر تاریک هوش مصنوعی یا (AI Winter) می گویند.
در این دوران بسیاری از پروژه ها تعطیل و سرمایه گذاری ها قطع شدند و دولت ها و سازمان های سرمایه گذار به دلیل عدم دستیابی به نتایج مطلوب از ادامه سرمایه گذاری منصرف شدند.

\subsection{عوامل اصلی عصر تاریک هوش مصنوعی}
\begin{itemize}
	\item \textbf{محدودیت‌های سخت‌افزاری:}
در آن زمان، سیستم‌های اولیه هوش مصنوعی  به محاسبات سنگینی نیاز داشتند که با توان پردازشی محدود آن زمان همخوانی نداشت.
	
	\item \textbf{کمبود داده‌ها:}
 در آن زمان، دسترسی به داده‌های کافی برای آموزش مدل‌های پیچیده ممکن نبود و الگوریتم‌های موجود به داده‌های بیشتری نیاز داشتند تا بتوانند به درستی آموزش ببینند و عملکرد مطلوبی داشته باشند.
	
	\item \textbf{روش‌های محدود یادگیری:} 
الگوریتم‌های اولیه به شدت به برنامه‌ریزی انسانی وابسته بودند و در بسیاری از موارد، مدل‌ها قادر به تعمیم به مسائل جدید نبودند و نمی توانستند تعمیم پذیری خیلی بالایی داشته باشند. \cite{russell2016artificial}.
	\end{itemize}

\subsection{پایان عصر تاریک و بازگشت هوش مصنوعی}
پس از چندین سال رکود و عدم سرمایه گذاری در حوزه هوش مصنوعی، سرانجام در دهه 1980 و 1990  عصر تاریک هوش مصنوعی با تحولات تکنولوژی و از همه مهم تر ظهور سیستم های خبره
(Expert-Systems)
به پایان رسید.
سیستم های خبره به عنوان یکی از اولین تلاش های موفق برای کاربردهای صنعتی در هوش مصنوعی به وجود آمدند.بر خلاف الگوریتم های اولیه، این سیستم هااز پایگاه بزرگ قواعد و قوانین 
(Rule-Based-Systems) 
استفاده میکردند. 
در سیستم های خبره به جای تلاش برای شبیه سازی کلی هوش مصنوعی، بر حل مسائل تخصصی برای صنایع و سازمان ها تمرکز میکردند. برای مثال، سیستم های خبره در پزشکی برای تشخیص بیماری ها و پیشنهاد درمان، در صنعت برای مدیریت و پیش بینی خرابی ماشین آلات، و در امور مالی برای تحلیل و ارزیابی ریسک کاربرد داشتند.
هر چند این سیستم ها نمی توانستند درک عمیق و هوشمندی عمومی را ایجاد کنند. اما برای رفع نیاز های پیچیده مناسب بودند.
همزمان با موفقیت این سیستم ها، بهبودهای زیادی در سخت افزارها و کاهش هزینه های پردازش به وجود آمد. در دهه های 1980  و 1990 کامپیوتر ها به تدریج قوی تر و مقرون به رفه تر شدند و امکان پردازش داده های بیشتر و اجرای الگوریتم های پیچیده تر فراهم شد.
این افزایش توان محاسباتی، نیاز به پردازش داده های بزرگ و پیچیده را فراهم کرد و در نتیجه دسترسی به داده ها و انجام محاسبات سنگین برای توسعه الگوریتم های جدید فراهم شد.
از طرف دیگر، پیشرفت های انجام شده در ذخیره سازی داده و رشد اینترنت باعث دسترسی گسترده تر به داده ها و منابع اطلاعاتی شد.
به این ترتیب، مجموعه ای از عوامل شامل ظهور سیستم های خبره، افزایش قدرت پردازش و دسترسی به داده های بیشتر، منجر به بازگشت هوش مصنوعی شد و این دوره نه تنها پایان عصر هوش مصنوعی بود، بلکه راه را برای الگوریتم های یادگیری ماشین و توسعه شبکه های عصبی هموار کرد. \cite{(McCorduck, 2004; Russell & Norvig, 2016).}



